\subsection{Phương pháp trình bày}

\textbf{Người phụ trách:} Huy.

\begin{itemize}
    \item Trình bày chi tiết về phương pháp gốc đã chọn. (đặt vấn đề, đề xuất phương pháp, tiến hành thực nghiệm, phân tích kết quả, bàn luận, tổng kết)
    \item Phân tích của nhóm về hạn chế tiềm ẩn của phương pháp.
\end{itemize}

Đồ án của nhóm được phát triển dựa trên nghiên cứu \textbf{Học biểu diễn không gian-thời gian phân cấp cho nhận dạng dáng đi}\cite{hstl}, viết tắt là \textbf{HSTL}. Lý do nhóm chọn nghiên cứu này là:
\begin{itemize}
    \item Được công bố năm 2023, là một trong những phương pháp SOTA tính tới thời điểm hiện tại (2025).
    \item Có mã nguồn mở từ chính tác giả.
    \item Tập dữ liệu dễ dàng truy cập.
    \item \textit{Lưu ý: đề tài \textbf{Nhận diện dáng đi} có nhiều nghiên cứu được công bố kèm với mã nguồn mở, nhưng hầu hết các bộ dữ liệu cho chủ đề này đều rất lớn (khoảng 100GB), không công khai mà phải yêu cầu quyền truy cập từ tác giả, đồng thời phải cung cấp nhiều thông tin không nằm trong khả năng của sinh viên trong môn học này. Cụ thể, hầu hết các tập dữ liệu bắt buộc nếu sinh viên muốn dùng, tác giả mặc định sinh viên sử dụng chúng trong các nhóm nghiên cứu của trường/khoa, nên phải có chữ ký của giảng viên, chủ nhiệm đề tài. Vì các thủ tục phức tạp và có thể tốn nhiều thời gian, gây ảnh hưởng tiến độ làm việc, nên nhóm quyết định không dùng các bộ dữ liệu giới hạn quyền truy cập kia, chỉ dùng duy nhất 1 tập công khai \textbf{CASIA-B} (trình bày sau).}
\end{itemize}

\subsubsection{Bối cảnh và vấn đề}

Các công nghệ liên quan đến sinh trắc học như vân tay, mống mắt, khuôn mặt đều yêu cầu dữ liệu được thu thập trong một điều kiện lý tưởng (ví dụ: ảnh khuôn mặt phải được chụp chính diện hoặc gần máy ảnh để xác định rõ định danh) và sự phối hợp giữa thiết bị với chủ thể (người cung cấp dữ liệu). Trong khi đó, dữ liệu dáng đi có thể được thu thập mà không cần sự phối hợp đó.

\subsubsection{Phương pháp đề xuất (gốc)}

\subsubsection{Cài đặt thực nghiệm}

\subsubsection{Kết quả}

\subsubsection{Bàn luận của tác giả}

\subsubsection{Phân tích của nhóm}
