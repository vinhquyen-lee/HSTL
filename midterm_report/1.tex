
\subsection{Tổng hợp nội dung chương sách được chọn}

% \textbf{Quyền phụ trách phần 1,4,5,P3D}\\

% \textbf{Mạnh phụ trách phần 2,3}

% Tổng hợp đầy đủ nội dung, với tổ chức thư mục \textbf{BẮT BUỘC} giống hoàn toàn của chương.
% \textbf{Lưu ý:}
% \begin{itemize}
%     \item Đề cập đầy đủ thông tin cốt lõi. Bỏ qua các \textit{kể chuyện dài dòng} (tóm tắt lại nếu cần).
%     \item Các biểu thức toán phải được viết bằng môi trường toán của \LaTeX.
%     \item Đề cập đầy đủ các số liệu.
%     \item Chèn hình ảnh nếu cần thiết.
%     \item \textcolor{red}{Không sao chép nội dung trong sách}.

% \end{itemize}

\subsubsection{Giới thiệu}

Với mục tiêu định danh con người từ khoảng cách xa và trong môi trường không bị ràng buộc, nhận diện dáng đi tập trung phân tích và xác định danh tính của cá nhân dựa trên các kiểu đi bộ đặc trưng hoặc cách thức vận động của họ qua các chuỗi hình ảnh được thu thập từ camera.
Khác với các đặc điểm sinh trắc học vật lý truyền thống như dấu vân tay, khuôn mặt hay mống mắt, dáng đi được xếp vào nhóm sinh trắc học hành vi. Đặc điểm này được hình thành từ cấu trúc cơ xương độc nhất và thói quen vận động của mỗi người, tạo nên một dấu ấn riêng biệt có khả năng phân biệt cao.
Ngay từ những năm 1970, các nghiên cứu tâm lý học nền tảng của Cutting và Kozlowski đã chứng minh rằng con người có khả năng nhận diện người quen chỉ thông qua sự chuyển động của các điểm sáng mô phỏng dáng đi mà không cần bất kỳ thông tin chi tiết nào về ngoại hình hay khuôn mặt.\\

Một trong những lợi thế quan trọng nhất giúp nhận diện dáng đi trở nên nổi bật là khả năng hoạt động hiệu quả trong các điều kiện khó khăn. Trong khi nhận diện khuôn mặt hay mống mắt yêu cầu đối tượng phải hợp tác, đứng gần thiết bị thu nhận và cần hình ảnh độ phân giải cao, nhận diện dáng
đi có thể thực hiện từ khoảng cách xa, có thể lên tới hàng trăm mét và chấp nhận dữ liệu có độ phân giải thấp. Đặc biệt, đây là một phương thức nhận diện không xâm lấn, nghĩa là hệ thống có thể thu thập dữ liệu thụ động qua camera giám sát mà không cần sự tương tác trực tiếp hay sự chủ động hợp tác
của đối tượng. Hơn nữa, do dáng đi là một hành vi vô thức bắt nguồn từ cơ chế sinh học, do đó việc ngụy trang hay giả mạo dáng đi trong thời gian dài là vô cùng khó khăn đối với các đối tượng muốn che giấu danh tính.

\begin{figure}[htbp]
    \centering
    \includegraphics[width=0.5\linewidth]{images/gait_recognition.jpg}
    \caption{Các loại dữ liệu cho bài toán nhận diện dáng đi.}
    \label{fig:gait_recognition}
\end{figure}

Nhìn chung, việc ứng dụng hệ thống nhận diện dáng đi hiện thực phải đối mặt với rất nhiều thử thách lớn đến từ các yếu tố ngoại cảnh. Độ chính xác của hệ thống thường sẽ rất dễ bị ảnh hưởng nghiêm trọng bởi sự thay đổi của góc nhìn của camera, đây được xem là yếu tố gây nhiễu lớn nhất làm
thay đổi hình dạng hình học của đối tượng trên khung hình. Bên cạnh đó, các điều kiện về trang phục như áo khoác dày che khuất cơ thể, hoặc trạng thái mang vác vật dụng như ba lô, túi xách cũng làm thay đổi trọng tâm và biên độ dao động của dáng đi. Các yếu tố môi trường khác như bề mặt đường đi,
điều kiện ánh sáng hay sự che khuất bởi vật cản cũng góp phần làm giảm hiệu suất nhận diện khi áp dụng vào các tình huống đời sống.
\subsubsection{Vấn đề}
Bài toán thử thách HumanID được thiết lập trong chương trình HumanID At a Distance của DARPA nhằm cung cấp một khung tham chiếu khách quan
và định lượng để đo lường tiến bộ trong nghiên cứu nhận dạng dáng đi. Bài toán bao gồm ba thành phần chính: cơ sở dữ liệu quy mô lớn, hệ thống
các thí nghiệm thử thách với độ khó tăng dần và một thuật toán cơ sở


\paragraph{a. Cơ sở dữ liệu và biến số}

Dữ liệu bao gồm 1870 chuỗi video từ 122 cá nhân, được thu thập trong môi trường ngoài trời.
Mỗi đối tượng được ghi hình dưới sự kết hợp của năm biến số chính:
\begin{itemize}
    \item \textbf{Góc quay camera:} Hai góc nhìn bên trái và bên phải.
    \item \textbf{Loại giày:} Hai loại giày khác nhau.
    \item \textbf{Bề mặt:} Cỏ và bê tông.
    \item \textbf{Vật dụng mang theo:} Có mang túi xách hoặc không mang.
    \item \textbf{Thời gian:} Hai thời điểm thu thập cách nhau 6 tháng.
\end{itemize}



Ngoài ra, bộ dữ liệu còn cung cấp các thông tin bổ trợ như nhân trắc học (tuổi, giới tính, chiều cao)
và các hình ảnh bóng được tách thủ công cho 71 đối tượng để hỗ trợ nghiên cứu lỗi phân đoạn và xây dựng mô hình chi tiết các bộ phận cơ thể.



\paragraph{b. Baseline Gait Algorithm}
Thuật toán cơ sở dựa trên việc khớp mẫu hình bóng. Quá trình này được mô tả thông qua các bước toán học sau:
\begin{enumerate}
    \item \textbf{Ước lượng chu kỳ:} Xác định chu kỳ dáng đi ($N_{Gait}$) dựa trên sự biến thiên định kỳ của số lượng pixel tiền cảnh ở phần dưới hình bóng. Gọi $f(t)$ là hàm đếm số pixel tại khung hình $t$, chu kỳ được tính dựa trên khoảng cách giữa các điểm cực tiểu liên tiếp của $f(t)$.



    \item \textbf{Độ đo tương quan không gian - thời gian:}
          Giả sử chuỗi kiểm tra là $S_P = \{S_P(1), \dots, S_P(M)\}$ và chuỗi chuẩn (Gallery) là $S_G = \{S_G(1), \dots, S_G(N)\}$. Quá trình so sánh thực hiện qua các bước:

          \begin{itemize}
              \item \textit{Độ tương đồng khung hình:} Sử dụng chỉ số Tanimoto ($S$) để so sánh hai khung hình $i$ và $j$:
                    \begin{equation}
                        S(S_P(i), S_G(j)) = \frac{\#(S_P(i) \cap S_G(j))}{\#(S_P(i) \cup S_G(j))}
                    \end{equation}
                    Trong đó, $\#$ là số lượng pixel tiền cảnh của phép giao ($\cap$) và phép hợp ($\cup$).

              \item \textit{Tương quan chuỗi con:} Chia $S_P$ thành các chuỗi con $S_{Pk}$ có độ dài $N_{Gait}$. Độ tương quan giữa $S_{Pk}$ và $S_G$ tại vị trí dịch chuyển $l$ là:
                    \begin{equation}
                        \text{Corr}(S_{Pk}, S_G)(l) = \sum_{j=1}^{N_{Gait}} S(S_P(k+j), S_G(l+j))
                    \end{equation}
          \end{itemize}

    \item \textbf{Hàm độ đo tương đồng cuối cùng:} Để chống nhiễu và sai số phân đoạn cục bộ, độ tương đồng tổng thể ($Sim$) được tính bằng giá trị trung vị của các giá trị tương quan cực đại:
          \begin{equation}
              Sim(S_P, S_G) = \text{Median}_k \left( \max_l \text{Corr}(S_{Pk}, S_G)(l) \right)
          \end{equation}
\end{enumerate}

\paragraph{c. Các thách thức trong thực nghiệm (Challenge Experiments)}

Để đánh giá hiệu suất của thuật toán dưới tác động của các biến số khác nhau, bài toán thiết lập 12 thách thức trong thí nghiệm. Một tập hợp chuẩn (\textit{Gallery}) được cố định làm nhóm đối chứng với các điều kiện: bề mặt cỏ, giày loại A, góc nhìn bên phải, không mang túi xách và dữ liệu thu thập vào tháng 5.

Các tập kiểm tra được thiết kế để cô lập hoặc kết hợp các biến số nhằm đo lường sự sụt giảm hiệu suất:

\begin{itemize} \item \textbf{Các thí nghiệm đơn biến :} Được đánh dấu () trong bảng thử thách, bao gồm: \begin{itemize} \item \textbf{Exp A (View):} Thay đổi góc nhìn (Trái vs Phải). \item \textbf{Exp B* (Shoe):} Thay đổi loại giày (A vs B). \item \textbf{Exp D* (Surface):} Thay đổi bề mặt đi bộ (Cỏ vs Bê tông). \item \textbf{Exp H* (Carry):} Thay đổi điều kiện mang vác (Có túi xách vs Không). \item \textbf{Exp K* (Time):} Thay đổi thời gian thu thập (Tháng 5 vs Tháng 11), bao gồm cả thay đổi ngầm định về trang phục. \end{itemize}

    \item \textbf{Các thí nghiệm đa biến:} Các thí nghiệm còn lại (như C, E, F, G, I, J, L) kết hợp từ hai đến ba biến số cùng lúc như là vừa thay đổi bề mặt, vừa thay đổi góc nhìn để kiểm tra tính bền vững của hệ thống trong các tình huống phức tạp hơn.
\end{itemize}

Việc phân cấp các thí nghiệm này cho phép xác định thứ tự độ khó của bài toán nhận dạng. Thông thường, các thí nghiệm liên quan đến bề mặt và thời gian được coi là những thử thách lớn nhất đối với các thuật toán nhận dạng dáng đi hiện nay.


\paragraph{d. Đánh giá hiệu suất }

Kết quả thực nghiệm trên bài toán thử thách HumanID phản ánh sự tiến bộ vượt bậc của các thuật toán nhận dạng dáng đi qua hai giai đoạn phát triển chính:

\begin{itemize} \item \textbf{Giai đoạn khởi đầu (2002):} Khi bài toán mới được công bố, thuật toán cơ sở (Baseline) đạt hiệu suất cao hơn hầu hết các thuật toán thế hệ đầu. Điều này cho thấy tính hiệu quả của phương pháp khớp mẫu hình bóng đơn giản nhưng ổn định.

    \item \textbf{Giai đoạn cải tiến (2004 -- 2006):} Hiệu suất của các thuật toán mới bắt đầu vượt xa Baseline một cách đáng kể. Sự cải thiện này không chỉ đến từ việc tối ưu hóa kỹ thuật (engineering) mà còn nhờ sự thay đổi trong tư duy tiếp cận: chuyển từ phân tích động lực học thuần túy sang phân tích hình dạng hình bóng (silhouette shapes) và các mô hình dựa trên bộ phận cơ thể.
\end{itemize}

Dựa trên các kết quả báo cáo, có thể rút ra các nhận định quan trọng sau: \begin{itemize} \item \textbf{Độ khó của biến số:} Các thí nghiệm thay đổi bề mặt (Exp D) và thời gian (Exp K) vẫn là những thách thức lớn nhất.
          Mặc dù hiệu suất đã được cải thiện, nhưng khoảng cách giữa kết quả thực tế và kỳ vọng vẫn còn lớn so với các thí nghiệm về góc nhìn hay loại giày. \item \textbf{Tính đa dạng của phương pháp:} Các nghiên cứu cho
          thấy động lực học là quan trọng nhưng chưa đủ; việc kết hợp thông tin về hình thái (morphology) giúp hệ thống bền vững hơn trước các sai số phân đoạn và nhiễu hậu cảnh. \end{itemize}

\subsubsection{Phương pháp}
Hầu hết các phương pháp nhận dạng dáng đi đều dựa trên việc khai thác hình bóng (silhouette) của đối tượng.
Đây là đặc trưng mức thấp phổ biến nhờ tính bền vững trước sự thay đổi màu sắc trang phục và khả năng trích xuất dễ dàng trong môi trường camera tĩnh.
Các hướng tiếp cận chính được chia thành ba nhóm:
\paragraph{a. Phương pháp dựa trên căn chỉnh thời gian}
Đây là hướng tiếp cận phổ biến nhất, coi chuỗi dáng đi là một chuỗi thời gian chứa đựng cả đặc trưng hình dạng và động lực học. Quá trình này thường bao gồm ba giai đoạn: trích xuất đặc trưng (silhouette, PCA, mô tả Fourier), căn chỉnh chuỗi và tính toán độ đo.
\begin{itemize}
    \item \textbf{Căn chỉnh chuỗi:} Sử dụng các kỹ thuật như Biến đổi thời gian động (Dynamic Time Warping - DTW) để đồng bộ hóa các chuỗi có tốc độ khác nhau, hoặc Mô hình Markov ẩn (HMM) để mô tả sự chuyển giao giữa các tư thế.
    \item \textbf{Mô hình toán học HMM:}
          \begin{equation}
              P(O|\lambda) = \sum_{q} \pi_{q_1} b_{q_1}(o_1) a_{q_1q_2} \dots b_{q_T}(o_T)
          \end{equation}
          \textbf{Trong đó:}
          \begin{itemize}
              \item $O = \{o_1, o_2, \dots, o_T\}$ là chuỗi các đặc trưng hình bóng quan sát được từ video.
              \item $\pi_{q_1}$: Xác suất bắt đầu tại trạng thái tư thế $q_1$.
              \item $a_{q_{t-1}q_t}$: Xác suất chuyển đổi giữa hai tư thế liên tiếp (ví dụ: từ tư thế chân chụm sang chân xòe). Nó mô tả \textbf{động lực học} của bước đi.
              \item $b_{q_t}(o_t)$: Xác suất để một hình bóng cụ thể $o_t$ xuất hiện tại trạng thái $q_t$. Nó mô tả \textbf{đặc điểm hình dạng} của tư thế đó.
              \item $P(O|\lambda)$ là tổng xác suất của chuỗi video khớp với mô hình mẫu $\lambda$ của một cá nhân.
          \end{itemize}
\end{itemize}
\paragraph{b. Phương pháp dựa trên hình dạng}Hướng tiếp cận này ưu tiên sự tương đồng về hình thái của hình bóng và giảm nhẹ yếu tố trình tự thời gian. Một số kỹ thuật tiêu biểu bao gồm:
\begin{itemize}
    \item \textbf{Hình ảnh năng lượng dáng đi (Gait Energy Image - GEI):}
          \begin{equation}
              A(x, y) = \frac{1}{N} \sum_{t=1}^{N} S(x, y, t)
          \end{equation}
          \textbf{Trong đó:}
          \begin{itemize}
              \item $S(x, y, t)$ là giá trị pixel (0 hoặc 1) tại tọa độ $(x, y)$ của khung hình thời điểm $t$.
              \item $A(x, y)$ (GEI) là một ảnh xám duy nhất đại diện cho toàn bộ chu kỳ.
              \item Các vùng có cường độ sáng cao nhất trong GEI cho thấy đó là nơi cơ thể ít thay đổi nhất (thường là đầu và thân), trong khi các vùng mờ (cường độ thấp) đại diện cho các bộ phận chuyển động mạnh như tay và chân.
          \end{itemize}
    \item \textbf{Chuẩn hóa động lực học :} Sử dụng một mô hình HMM chung cho toàn bộ quần thể để ánh xạ các khung hình vào các khung hình tư thế chuẩn (\textit{stance-frames}). Khoảng cách giữa hai đối tượng sau đó được tính toán trong không gian.
    \item \textbf{Phân tích biệt thức tuyến tính (LDA):}
          \begin{equation}
              J(W) = \frac{|W^T S_B W|}{|W^T S_W W|}
          \end{equation}
          \textbf{Trong đó:}
          \begin{itemize}
              \item $S_B$: Đo lường sự khác biệt giữa các cá nhân khác nhau. Chúng ta muốn giá trị này lớn nhất để dễ phân biệt người này với người kia.
              \item $S_W$: Đo lường sự biến thiên của cùng một người dưới các điều kiện khác nhau (ví dụ: cùng một người nhưng khi mặc áo khoác hoặc đi trên cỏ).
          \end{itemize}
          \paragraph{c. Phương pháp tham số tĩnh }Hướng tiếp cận này trích xuất các thông số hình học trực tiếp từ cơ thể và các đặc trưng vận động cơ bản:
          \begin{itemize}
              \item \textbf{Tham số vận động:} Độ dài sải chân , tốc độ bước đi và nhịp độ .
              \item \textbf{Tham số hình thể:} Tỷ lệ kích thước các bộ phận cơ thể (tỷ lệ chiều dài chân/thân).
          \end{itemize}Tuy nhiên, phương pháp này thường đòi hỏi việc hiệu chuẩn camera 3D phức tạp và có hiệu suất thấp hơn khi xử lý dữ liệu ở khoảng cách xa hoặc độ phân giải thấp.
\end{itemize}
\subsubsection{Thảo luận và Hướng nghiên cứu tiếp theo}

\textbf{Hình dáng và động lực học của dáng đi}

Trong nghiên cứu về dáng đi, sự tranh luận giữa vai trò của hình dáng và động lực học luôn là tiêu điểm. Các nghiên cứu thực nghiệm ban đầu chỉ ra rằng con người có khả năng nhận diện danh tính dựa trên các đặc điểm động lực học ngay cả khi hình dáng bị che khuất, tuy nhiên các phương pháp phân
tích dựa trên hình dáng hình bóng lại mang lại hiệu quả vượt trội. Nhiều thuật toán tiên tiến gần đây đã chứng minh rằng việc tập trung vào hình dáng của từng giai đoạn trong chu kỳ bước đi giúp hệ thống đạt được độ chính xác cao hơn, đặc biệt là khi phải đối mặt với các biến số khó như thay đổi
bề mặt đi bộ. Mặc dù vậy, động lực học vẫn đóng vai trò không thể thiếu vì nó chứa đựng các thông tin về tốc độ và sự chuyển tiếp giữa các pha vận động vốn mang tính đặc trưng cho từng cá nhân. Tuy nhiên, các nghiên cứu mới gần đây chỉ ra rằng việc phụ thuộc quá nhiều vào hình dáng sẽ khiến hệ
thống dễ bị sai lệch khi đối tượng thay đổi trang phục. Do đó, xu hướng hiện tại là phát triển các mô hình học biểu diễn không gian - thời gian phân cấp, cho phép tách biệt các đặc trưng chuyển động cốt lõi ra khỏi các đặc điểm hình dáng bề ngoài dễ thay đổi, từ đó tận dụng sức mạnh của cả hai yếu tố này.

Hình \ref{fig:gait_cycle} minh họa chu kỳ dáng đi được chia thành các giai đoạn khác nhau, thể hiện rõ sự chuyển động tuần hoàn của các bộ phận cơ thể trong quá trình đi bộ.

\begin{figure}[htbp]
    \centering
    \includegraphics[width=1\linewidth]{images/gait_cycle.jpg}
    \caption{Minh họa chu kỳ dáng đi được chia thành bốn giai đoạn: (i) tựa chân phải; (ii) lăng chân trái; (iii) tựa chân trái; và (iv) lăng chân phải, tương ứng với các trạng thái từ (a) đến (e). Khoảng thời gian cả hai chân cùng tiếp xúc mặt sàn được gọi là giai đoạn hỗ trợ kép.}
    \label{fig:gait_cycle}
\end{figure}

\textbf{Chất lượng hình bóng và nhận dạng dáng đi}

Chất lượng của các hình bóng phụ thuộc vào khả năng phân biệt giữa nền và đối đượng. Trong môi trường ngoài trời, các yếu tố nhiễu như bóng đổ, thay đổi ánh sáng và sự chuyển động của hậu cảnh khiến việc tách hình bóng trở nên cực kỳ khó khăn. Tuy nhiên, một phát hiện chỉ ra rằng sự sụt giảm
hiệu suất khi thay đổi bề mặt hoặc thời gian không hoàn toàn do lỗi xử lý hình bóng ở cấp độ thấp gây ra. Ngay cả khi sử dụng các hình bóng đã được tiền xử lý thủ công, kết quả vẫn cho thấy sự suy giảm đáng kể, nghĩa là bản thân dáng đi của con người đã có những thay đổi cơ bản khi điều kiện
môi trường thay đổi. Điều này cho thấy các công trình tương tai thay vì tìm kiếm các phương pháp tốt hơn để phát hiện hình bóng nhằm cải thiện nhận dạng, việc nghiên cứu và tách biệt các thành phần của dáng đi không thay đổi theo giày dép, bề mặt hoặc thời gian sẽ hiệu quả hơn.

\textbf{Các biến số}

Thách thức của nhận diện dáng đi nằm ở việc duy trì độ ổn định trước các tác động của các yếu tố ngoại cảnh. Trong khi các yếu tố như loại giày dép hay việc mang theo túi xách có tác động tương đối nhỏ, thì sự thay đổi về bề mặt đi bộ và khoảng cách thời gian giữa các lần thu thập dữ
liệu lại gây ra những ảnh hưởng tiêu cực nghiêm trọng. Đặc biệt, nhận dạng dáng đi theo thời gian là một bài toán khó do sự thay đổi về trang phục theo mùa và những biến đổi tự nhiên trong cơ thể người theo thời gian. Ngoài ra, việc di chuyển trong các môi trường thực tế với các góc nhìn
camera đa dạng, vật cản che khuất và độ phân giải thấp vẫn là những trở ngại cần được giải quyết. Do đó các nghiên cứu trong tương lai cần tập trung vào việc mô hình hóa các thành phần dáng đi không thay đổi hoặc tìm cách dự đoán sự biến đổi của dáng đi khi chuyển từ bề mặt này sang bề mặt khác.

\textbf{Các bộ dữ liệu trong tương lai}

Sự phát triển của nhận dạng dáng đi phụ thuộc rất lớn vào quy mô và độ đa dạng của các bộ dữ liệu. Các chuyên gia nhận định rằng cần có những bộ dữ liệu khổng lồ với quy mô lên tới hàng nghìn đối tượng để hỗ trợ việc hiểu sâu hơn về sự biến thiên của dáng đi trong điều kiện ngoài trời và theo thời
gian. Các bộ dữ liệu chuẩn như CASIA-B hay OU-MVLP đang dần trở nên bão hòa khi các mô hình Học sâu đã đạt độ chính xác rất cao trên đó. Hiện nay, các bộ dữ liệu như GREW hay Gait3D đã bắt đầu chuyển hướng từ môi trường phòng thí nghiệm sang môi trường thực tế với hàng chục nghìn danh tính và hàng
triệu chuỗi hình ảnh. Một hướng đi mới đầy triển vọng là sử dụng dữ liệu tổng hợp được tạo ra từ các mô hình cơ thể người ảo, giúp giải quyết vấn đề thiếu hụt dữ liệu được dán nhãn và giảm bớt các rào cản về chi phí thu thập dữ liệu thực tế. Đồng thời, việc khai thác dữ liệu video không nhãn trên
quy mô lớn thông qua các phương pháp học tự giám sát đang trở thành một lĩnh vực nghiên cứu đầy tiềm năng.

\textbf{Kết hợp khuôn mặt và dáng đi}

Mặc dù dáng đi có ưu thế ở khoảng cách xa, thì việc kết hợp giữa nhận dạng dáng đi và khuôn mặt là một giải pháp tối ưu để nâng cao độ tin cậy của hệ thống. Dáng đi có thể được sử dụng khi đối tượng ở khoảng cách xa, trong khi khuôn mặt sẽ phát huy tác dụng khi đối tượng tiến lại gần camera hơn.
Việc kết hợp đa phương thức dáng đi và mặt người mang lại hiệu suất vượt trội so với việc chỉ sử dụng một loại sinh trắc học đơn lẻ, đồng thời giúp hệ thống bền bỉ hơn trước các nhiễu động của từng loại dữ liệu. Hình \ref{fig:face_samples} minh họa các mẫu khuôn mặt dưới nhiều điều kiện khác nhau, cho thấy sự đa dạng trong dữ liệu sinh trắc học. Trong tương lai, việc tích hợp này không chỉ dừng lại ở khuôn mặt mà còn có thể mở
rộng sang các phương thức khác như thông tin về chiều cao, kích thước các bộ phận cơ thể hoặc dữ liệu từ nhiều góc nhìn camera cùng lúc để tạo ra một hồ sơ định danh toàn diện và chính xác hơn.

\begin{figure}[htbp]
    \centering
    \includegraphics[width=0.8\linewidth]{images/face_samples.jpg}
    \caption{Các mẫu khuôn mặt dưới nhiều điều kiện khác nhau: (a) và (b) là ảnh tập mẫu trong điều kiện ánh sáng khác nhau; (c) và (d) là ảnh tập kiểm tra chụp ngoài trời ở khoảng cách xa và gần.}
    \label{fig:face_samples}
\end{figure}

\textbf{Quyền riêng tư và bảo mật sinh trắc học}

Một khía cạnh mới đang ngày càng được chú trọng là tính bảo mật và quyền riêng tư của dữ liệu dáng đi. Do dáng đi có thể được thu thập từ xa mà không cần sự hợp tác hay nhận biết của đối tượng, công nghệ này làm dấy lên những lo ngại nghiêm trọng về quyền riêng tư cá nhân và sự giám sát đại
chúng. Bên cạnh đó, vấn đề an ninh của chính hệ thống AI cũng đáng báo động với sự xuất hiện của các cuộc tấn công đối kháng, nơi kẻ tấn công có thể thay đổi một chút dáng đi hoặc thêm nhiễu vào video để đánh lừa hệ thống. Các luật lệ như quy định chung về Bảo vệ Dữ liệu Châu Âu (GDPR) đã
đặt ra những hạn chế chặt chẽ đối với việc sử dụng dữ liệu sinh trắc học, thúc đẩy cộng đồng nghiên cứu phải tìm kiếm các giải pháp bảo vệ quyền riêng tư. Các hướng nghiên cứu mới bao gồm việc phát triển các phương pháp ẩn danh hóa, mã hóa video dáng đi sao cho hệ thống nhận dạng vẫn hoạt
động nhưng danh tính con người không thể bị quan sát bằng mắt thường, cũng như tăng cường khả năng chống lại các cuộc tấn công giả mạo hoặc tấn công đối nghịch nhằm đánh lừa hệ thống nhận dạng.

\subsubsection{Kết luận}

Nhìn lại toàn bộ quá trình phát triển, nhận dạng dáng đi đã khẳng định vị thế là một trong những công nghệ sinh trắc học tiềm năng nhất, với điểm mạnh về khả năng định danh tầm xa và không xâm lấn mà các phương pháp truyền thống như khuôn mặt hay vân tay không thể thay thế.
Sự chuyển dịch mạnh mẽ từ các kỹ thuật thị giác máy tính cổ điển sang các kiến trúc Học sâu tiên tiến đã nâng tầm lĩnh vực này, giúp các hệ thống hiện đại đạt được độ chính xác ấn tượng trên các bộ dữ liệu tiêu chuẩn. Các nghiên cứu đột phá gần đây đã chứng minh rằng việc khai thác sâu các biểu diễn
không gian - thời gian phân cấp là chìa khóa để tách biệt các đặc trưng vận động cốt lõi khỏi những yếu tố nhiễu loạn của bề mặt, mở ra triển vọng to lớn trong việc giải mã hành vi vận động của con người.

Tuy nhiên, thực tế triển khai đã chỉ ra một khoảng cách đáng kể về tính thực tiễn giữa môi trường phòng thí nghiệm lý tưởng và thế giới thực hỗn loạn. Như một vài phân tích thực nghiệm đã làm rõ hiệu suất của các thuật toán hàng đầu vẫn sụt giảm nghiêm trọng khi đối mặt với sự đa dạng không giới hạn
của các biến số ngoại cảnh như góc quay camera an ninh phức tạp, sự thay đổi trang phục theo mùa, hay điều kiện ánh sáng và vật che khuất trong môi trường tự nhiên. Điều này khẳng định rằng, mặc dù chúng ta đã giải quyết tốt bài toán so khớp mẫu trong điều kiện kiểm soát, nhưng bài toán nhận dạng ở
ngoài thực tế vẫn là thách thức lớn cần tiếp tục giải quyết.

Trong tương lai, sự phát triển của nhận dạng dáng đi sẽ không còn đơn thuần là cuộc đua về các chỉ số độ chính xác trên tập dữ liệu cũ, mà sẽ là sự chuyển mình sang các hệ thống thông minh, bền vững và an toàn hơn. Xu hướng tất yếu sẽ là sự kết hợp đa phương thức giữa dữ liệu để vượt qua giới hạn của
camera quang học, cùng với việc áp dụng các kỹ thuật học không giám sát để tận dụng nguồn dữ liệu khổng lồ chưa gán nhãn. Đồng thời, khi công nghệ này đi sâu vào đời sống, các vấn đề về bảo mật chống giả mạo và bảo tồn quyền riêng tư sẽ trở thành những trụ cột quan trọng ngang hàng với hiệu năng kỹ thuật,
đảm bảo rằng nhận dạng dáng đi không chỉ là một công cụ giám sát hiệu quả mà còn là một công nghệ có trách nhiệm và đáng tin cậy.
