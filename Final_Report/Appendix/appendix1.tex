\appendix

\chapter{Phụ lục}
\label{Appendix1}

% ===================================================================
% VÍ DỤ VỀ PHỤ LỤC
% ===================================================================

\section{Bảng đối chiếu thuật ngữ}

Bảng \ref{tab:vietanh} cung cấp bảng đối chiếu các thuật ngữ Việt-Anh:

\begin{table}[ht]
    \centering
    \caption{Phụ lục đối chiếu Việt Anh}
    \label{tab:vietanh}
    \begin{tabular}{|p{7cm}|p{7cm}|}
        \hline
        \textbf{Tiếng Việt} & \textbf{Tiếng Anh} \\
        \hline
        Trí tuệ nhân tạo & Artificial Intelligence \\
        Học sâu & Deep Learning \\
        Mạng nơ-ron tích chập & Convolutional Neural Network \\
        Mạng nơ-ron & Neural Network \\
        Bộ mã hóa & Encoder \\
        Bộ giải mã & Decoder \\
        Chú ý & Attention \\
        Chú ý chéo & Cross-Attention \\
        Bộ biến đổi & Transformer \\
        Mô hình khuếch tán & Diffusion Model \\
        \hline
    \end{tabular}
\end{table}

\section{Mã nguồn mẫu}

Bạn có thể chèn mã nguồn vào phụ lục:

\begin{lstlisting}[language=Python, caption=Ví dụ mã Python]
def hello_world():
    print("Hello, World!")
    return True

if __name__ == "__main__":
    hello_world()
\end{lstlisting}

\section{Thông tin bổ sung}

Phụ lục có thể chứa các thông tin bổ sung như:
\begin{itemize}
    \item Dữ liệu thực nghiệm chi tiết
    \item Mã nguồn đầy đủ
    \item Các bảng kết quả bổ sung
    \item Tài liệu tham khảo bổ sung
\end{itemize}

% ===================================================================
% LƯU Ý VỀ PHỤ LỤC:
% - Sử dụng \appendix để bắt đầu phần phụ lục
% - Các chương trong phụ lục được đánh số tự động (Phụ lục A, B, ...)
% - Có thể sử dụng \chapter hoặc \section tùy theo cấu trúc
% - Phụ lục thường chứa thông tin bổ sung, không bắt buộc phải đọc
% ===================================================================
