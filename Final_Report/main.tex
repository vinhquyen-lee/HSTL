% ===================================================================
% TEMPLATE BÁO CÁO KHÓA LUẬN TỐT NGHIỆP
% Trường Đại học Khoa học Tự nhiên - ĐHQG TP.HCM
% Khoa Công nghệ Thông tin
% ===================================================================
%
% Template này cung cấp cấu trúc và các ví dụ về:
% - Cách trích dẫn tài liệu tham khảo
% - Cách chèn hình ảnh với các tùy chỉnh
% - Cách tạo bảng trong các trường hợp khác nhau
% - Cách viết công thức toán học và ký hiệu
% - Cách điều chỉnh khoảng cách dòng và văn bản
%
% HƯỚNG DẪN SỬ DỤNG:
% 1. Đọc file README.md để biết cách sử dụng template
% 2. Điền thông tin cá nhân vào các biến ở dòng 139-143
% 3. Thay thế nội dung mẫu trong các chapter bằng nội dung thực tế
% 4. Build theo thứ tự: BIB > PDF > PDF (hoặc sử dụng LaTeX build chain)
% ===================================================================

\documentclass[oneside,a4paper,14pt]{extreport}

% Font tiếng Việt
\usepackage[T5]{fontenc}
\usepackage[utf8]{inputenc}
\usepackage[utf8]{vietnam}
\DeclareTextSymbolDefault{\DH}{T1}

% Tài liệu tham khảo
%\usepackage[
%	sorting=nty,
%	backend=biber,
%    style=ieee
%	defernumbers=true]{biblatex}
\usepackage[style=ieee]{biblatex}
\usepackage[unicode]{hyperref} % Bookmark tiếng Việt
\addbibresource{References/references.bib}
%\bibliographystyle{IEEEtran}
%\bibliography{IEEEabrv,references}
\usepackage{nameref} % reference the name


\makeatletter
\def\blx@maxline{77}
\makeatother

% Chèn hình, các hình trong luận văn được để trong thư mục Images/
\usepackage{graphicx}
\graphicspath{ {figures/} }

% Chèn và định dạng mã nguồn
\usepackage{listings}
\usepackage{color}
\definecolor{codegreen}{rgb}{0,0.6,0}
\definecolor{codegray}{rgb}{0.5,0.5,0.5}
\definecolor{codepurple}{rgb}{0.58,0,0.82}
\definecolor{backcolour}{rgb}{0.95,0.95,0.92}
\lstdefinestyle{mystyle}{
    backgroundcolor=\color{backcolour},
    commentstyle=\color{codegreen},
    keywordstyle=\color{magenta},
    numberstyle=\tiny\color{codegray},
    stringstyle=\color{codepurple},
    basicstyle=\footnotesize,
    breakatwhitespace=false,
    breaklines=true,
    captionpos=b,
    keepspaces=true,
    numbers=left,
    numbersep=5pt,
    showspaces=false,
    showstringspaces=false,
    showtabs=false,
    tabsize=2
}
\lstset{style=mystyle}

% Chèn và định dạng mã giả
\usepackage{amsmath}
\usepackage{amssymb}
\usepackage{bm} % Only include this if you need bold math symbols
\usepackage{algorithm}
\usepackage[noend]{algpseudocode}
\makeatletter
\def\BState{\State\hskip-\ALG@thistlm}
\makeatother

% Đổi tên các câu lệnh toán học
\newcommand{\mb}{\mathbf}
\newcommand{\mc}{\mathcal}
\newcommand{\ptraj}{q\left( \mb x^{(0\cdots T)} \right)}
\newcommand{\ptrajtil}{\tilde{q}\left( \mb x^{(0\cdots T)} \right)}
\newcommand{\qtrajtil}{\tilde{p}\left( \mb x^{(0\cdots T)} \right)}
\newcommand{\pcondtraj}{q\left( \mb x^{(1\cdots T)} | \mb x^{(0)} \right)}
\newcommand{\qtraj}{p\left( \mb x^{(0\cdots T)} \right)}
\newcommand{\ptarget}{\pi\left( \mb x^{(T)} \right)}
\newcommand{\qmarg}{p\left( \mb x^{(0)} \right)}
\newcommand{\pmarg}{q\left( \mb x^{(T)} \right)}
% Bảng biểu
\usepackage{multirow}
\usepackage{array}
\newcolumntype{L}[1]{>{\raggedright\let\newline\\\arraybackslash\hspace{0pt}}m{#1}}
\newcolumntype{C}[1]{>{\centering\let\newline\\\arraybackslash\hspace{0pt}}m{#1}}
\newcolumntype{R}[1]{>{\raggedleft\let\newline\\\arraybackslash\hspace{0pt}}m{#1}}
\usepackage{booktabs}

% Vẽ sơ đồ
\usepackage{tikz}
\usetikzlibrary{positioning, arrows.meta, shapes.geometric, calc, fit, backgrounds,patterns,patterns.meta}

% Đổi tên mặc định
\renewcommand{\chaptername}{Chương}
\renewcommand{\figurename}{Hình}
\renewcommand{\tablename}{Bảng}
\renewcommand{\contentsname}{Mục lục}
\renewcommand{\listfigurename}{Danh sách hình}
\renewcommand{\listtablename}{Danh sách bảng}
\renewcommand{\appendixname}{Phụ lục}


% Kích thước Chapter
\usepackage{titlesec}
\titleformat{\chapter}[display]
  {\LARGE\bfseries}
  {\chaptertitlename\ \thechapter}{18pt}{\huge}


% Dãn dòng 1.5
\usepackage{setspace}
\onehalfspacing

% Thụt vào đầu dòng
\usepackage{indentfirst}

% Canh lề
\usepackage[
  top=30mm,
  bottom=25mm,
  left=30mm,
  right=20mm,
  includefoot]{geometry}

% Trang bìa
\usepackage{tikz}
\usetikzlibrary{calc}
\newcommand\HRule{\rule{\textwidth}{1pt}}

% ========================================================================================= %
% THÔNG TIN CÁ NHÂN - ĐIỀN VÀO ĐÂY                                                         %
% ========================================================================================= %
% Lưu ý: Dấu ~ là khoảng trắng không được tách (các chữ nối với nhau bằng dấu ~ sẽ nằm cùng 1 dòng)
\newcommand{\tenSV}{Phạm~Thái~Huy~-~Nguyễn~Đức~Mạnh~-~Lê~Quang~Vĩnh~Quyền} % Ví dụ: Nguyễn~Văn~A~-~Trần~Thị~B
\newcommand{\mssv}{21120081~22120204~22120307} % Ví dụ: 21120001~21120002
\newcommand{\tenKL}{CẢI TIẾN\\NGHIÊN CỨU HỌC BIỂU DIỄN\\KHÔNG GIAN-THỜI GIAN PHÂN CẤP\\CHO BÀI TOÁN NHẬN DẠNG DÁNG ĐI} % Ví dụ: Ứng~dụng~học~sâu~trong~xử~lý~ảnh
\newcommand{\tenGVHD}{PGS.TS.~Lê~Hoàng~Thái\\ThS.~Dương~Thái~Bảo} % Ví dụ: PGS.TS.~Nguyễn~Văn~X
\newcommand{\tenBM}{Tên~Bộ~Môn} % Ví dụ: Trí~tuệ~nhân~tạo

\begin{document}

\begin{titlepage}

  \begin{center}
    %ĐẠI HỌC QUỐC GIA THÀNH PHỐ HỒ CHÍ MINH\\
    TRƯỜNG ĐẠI HỌC KHOA HỌC TỰ NHIÊN\\
    \textbf{KHOA CÔNG NGHỆ THÔNG TIN}\\[1.8cm]


    { \Large \bfseries \tenSV\\[1.8cm] }

    %Tên đề tài Khóa luận tốt nghiệp/Đồ án tốt nghiệp

    { \Large \bfseries \tenKL \\[3cm]}


    %Chọn trong các dòng sau
    \large BÁO CÁO CUỐI KỲ - MÔN SINH TRẮC HỌC\\
    %\large ĐỒ ÁN TỐT NGHIỆP CỬ NHÂN\\
    %\large THỰC TẬP TỐT NGHIỆP CỬ NHÂN\\
    %Đưa vào dòng này nếu thuộc chương trình Chất lượng cao, hoặc lớp Cử nhân tài năng
    \large CHƯƠNG TRÌNH CHUẨN\\
    %\large CHƯƠNG TRÌNH CHẤT LƯỢNG CAO\\
    %\large CHƯƠNG TRÌNH CỬ NHÂN TÀI NĂNG\\[1.8cm]


    \begin{tikzpicture}[remember picture, overlay]
      \draw[line width = 2pt] ($(current page.north west) + (2cm,-2cm)$) rectangle ($(current page.south east) + (-1.5cm,2cm)$);
    \end{tikzpicture}

    \vfill
    Tp. Hồ Chí Minh, \today % Mot nho sua lai tieng Viet

  \end{center}

  \pagebreak



  \begin{center}

    TRƯỜNG ĐẠI HỌC KHOA HỌC TỰ NHIÊN\\
    \textbf{KHOA CÔNG NGHỆ THÔNG TIN}\\[1.8cm]


    % {\large \bfseries \tenSV\\}
    % {\large \bfseries \mssv\\[1.8cm]}

    {\large \bfseries Phạm Thái Huy - 21120081\\}
    {\large \bfseries Nguyễn Đức Mạnh - 22120204\\}
    {\large \bfseries Lê Quang Vĩnh Quyền - 22120307\\[1.8cm]}


    %Tên đề tài Khóa luận tốt nghiệp/Đồ án tốt nghiệp

    { \Large \bfseries \tenKL\\[1.8cm] }


    %Chọn trong các dòng sau
    \large BÁO CÁO CUỐI KỲ - MÔN SINH TRẮC HỌC\\
    %\large ĐỒ ÁN TỐT NGHIỆP CỬ NHÂN\\
    %Đưa vào dòng này nếu thuộc chương trình Chất lượng cao, hoặc lớp Cử nhân tài năng
    \large CHƯƠNG TRÌNH CHUẨN\\[1.8cm]
    %\large CHƯƠNG TRÌNH CHẤT LƯỢNG CAO\\[1.8cm]
    %\large CHƯƠNG TRÌNH CỬ NHÂN TÀI NĂNG\\[1.8cm]

    \textbf{GIẢNG VIÊN HƯỚNG DẪN}\\
    \tenGVHD\\

    \begin{tikzpicture}[remember picture, overlay]
      \draw[line width = 2pt] ($(current page.north west) + (2cm,-2cm)$) rectangle ($(current page.south east) + (-1.5cm,2cm)$);
    \end{tikzpicture}

    \vfill
    Tp. Hồ Chí Minh, \today

  \end{center}
  \pagenumbering{gobble}
\end{titlepage}

% Sasu trang Title, các bạn chèn nhận xét gủa GVHD và GVPB. Nhận xét sẽ được giáo vụ phát sau buổi bảo vệ để các bạn đóng quyển.

\cleardoublepage

\pagenumbering{roman} % Đánh số i, ii, iii, ...


\phantomsection
%\addcontentsline{toc}{chapter}{Lời cam đoan}
%\chapter*{Lời cam đoan}
\label{reassurances}


\chapter*{Lời cam đoan}
\label{commitment}

Chúng tôi xin cam đoan rằng đây là công trình nghiên cứu của chính chúng tôi. Dữ liệu và kết quả nghiên cứu trình bày trong báo cáo khóa luận là trung thực và không trùng lặp với bất kỳ đề tài nào khác. Chúng tôi cam kết rằng toàn bộ nội dung của báo cáo này không sao chép hay đạo văn từ bất kỳ công trình nghiên cứu nào khác.

Trong suốt quá trình thực hiện khóa luận, chúng tôi luôn chấp hành đầy đủ các quy định về đạo đức nghiên cứu, đảm bảo trích dẫn đúng và đầy đủ tất cả các nguồn tài liệu tham khảo, đồng thời giữ vững tinh thần trung thực học thuật trong mọi khía cạnh của báo cáo.


%\addcontentsline{toc}{chapter}{Thuyết minh chỉnh sửa đề tài}
%\chapter*{Thuyết minh chỉnh sửa đề tài}
\label{edit}

Theo mẫu thuyết minh chỉnh sửa đề tài (nếu có thay đổi tên đề tài, nội dung báo cáo... trong cuốn báo cáo sau bảo vệ)

%\addcontentsline{toc}{chapter}{Nhận xét của GV hướng dẫn}
%\chapter*{Nhận xét hướng dẫn}
\label{advisor}

Theo bản nhận xét của giảng viên hướng dẫn (có chữ kí) do giáo vụ cung cấp.

%\addcontentsline{toc}{chapter}{Nhận xét của GV phản biện}
%\chapter*{Nhận xét phản biện}
\label{reviewer}

Theo bản nhận xét của giảng viên phản biện (có chữ kí) do giáo vụ cung cấp.

\addcontentsline{toc}{chapter}{Lời cảm ơn}
\chapter*{Lời cảm ơn}
\label{thanks}
\textcolor{blue}{Có cần không ?\\}
Trước hết, chúng tôi xin bày tỏ lòng biết ơn chân thành và sâu sắc đến PGS.TS. Lê Hoàng Thái và ThS. Dương Thái Bảo vì sự hướng dẫn tận tình, khích lệ và hỗ trợ quý báu, giúp chúng tôi có cơ hội khám phá các hướng nghiên cứu mới.

Trong suốt quá trình thực hiện đề tài, quý Thầy đã chia sẻ những kiến thức giá trị và đóng góp nhiều gợi ý sâu sắc, giúp chúng tôi định hướng được con đường nghiên cứu của riêng nhóm thông qua các buổi trao đổi từ những ngày đầu cho đến khi hoàn thành nghiên cứu.

% Chúng tôi cũng xin gửi lời cảm ơn chân thành đến toàn thể quý Thầy Cô của Khoa Công nghệ thông tin nói riêng và quý Thầy Cô của Trường Đại học Khoa học tự nhiên, Đại học Quốc gia TP. Hồ Chí Minh nói chung vì sự tận tâm giảng dạy và chia sẻ kinh nghiệm quý báu trong suốt những năm học vừa qua.

% Cuối cùng, chúng tôi xin gửi lời tri ân sâu sắc nhất đến gia đình vì đã luôn bên cạnh, động viên và tiếp thêm sức mạnh cho chúng tôi vượt qua mọi khó khăn.


% \addcontentsline{toc}{chapter}{Đề cương}
% %Đây là template dùng cho đề cương đề tài tốt nghiệp
%Khoa Công nghệ Thông tin
%Trường Đại học Khoa học Tự nhiên, ĐHQG-HCM

%Liên hệ về mẫu LaTEX này: Thầy Bùi Huy Thông (bhthong@fit.hcmus.edu.vn)

\documentclass{article}[14pt]
\usepackage[utf8]{vietnam}
\usepackage{enumerate}
\usepackage{enumitem}
\usepackage{multicol}
\usepackage{multirow}
\usepackage{makecell}
\usepackage{listings}
\usepackage[left=2cm,right=2cm,top=2.5cm,bottom=2.5cm]{geometry}
\usepackage{verbatim}
\usepackage{graphicx}
\usepackage{url}
\usepackage{fancyhdr}
\usepackage{fancybox,framed}
\linespread{1.3}
\usepackage{lastpage}
\usepackage{floatrow}
\pagenumbering{arabic}
\usepackage{amsmath}  
\usepackage{amssymb}
\usepackage{cite}

% % Setup fancy headers and footers - no need at the moment
% \pagestyle{fancy}
% \fancyhf{} % Clear all header and footer fields
% \fancyhead[L]{Đề cương khóa luận tốt nghiệp}
% \fancyhead[R]{\thepage}
% \fancyfoot[L]{Phạm Thái Huy - Tiêu Ân Tuấn}
% \fancyfoot[C]{Tô màu ảnh độ xám}
% \fancyfoot[R]{Trang \thepage\ của \pageref{LastPage}}
% \renewcommand{\headrulewidth}{0.4pt}
% \renewcommand{\footrulewidth}{0.4pt}

\newfloatcommand{capbtabbox}{table}[][\FBwidth]
\usepackage{longtable}
\usepackage{array}
\usepackage{blindtext}
\usepackage{titlesec}
\usepackage[nottoc]{tocbibind}
\usepackage[unicode]{hyperref}
\usepackage{indentfirst}
% no need at the moment
% \usepackage[unicode,
%     bookmarks=true,
%     bookmarksnumbered=true,
%     bookmarksopen=true,
%     colorlinks=true,
%     linkcolor=black,
%     citecolor=blue,
%     urlcolor=blue,
%     pdftitle={Đề cương khóa luận: Tô màu ảnh độ xám},
%     pdfauthor={Phạm Thái Huy, Tiêu Ân Tuấn},
%     pdfsubject={Khóa luận tốt nghiệp},
%     pdfkeywords={tô màu ảnh, độ xám, học sâu, GAN}
% ]{hyperref}

\titleformat*{\section}{\LARGE\bfseries}
\titleformat*{\subsection}{\Large\bfseries}
\titleformat*{\subsubsection}{\large\bfseries}
%\addbibresource{ref.bib}

\begin{document}
\begin{figure}[h]
    \begin{floatrow}
        \ffigbox{\includegraphics[scale = .4]{logo.png}}
        {%

        }
        \capbtabbox{
            \begin{tabular}{l}
                \multicolumn{1}{c}{\textbf{\begin{tabular}[c]{@{}c@{}}TRƯỜNG ĐẠI HỌC KHOA HỌC TỰ NHIÊN\\KHOA CÔNG NGHỆ THÔNG TIN\end{tabular}}} \\ \\ \\
            \end{tabular}
        }
        {%

        }
    \end{floatrow}
\end{figure}

\begin{center}

    %Xác định loại đề tài tốt nghiệp tương ứng: Khóa luận, Thực tập, Đồ án
    \textbf{\Large BÁO CÁO TIẾN ĐỘ GIỮA KỲ} \\
\end{center}

%\vspace{.5cm}

\begin{center}
    %Tên đề tài phải VIẾT HOA

    \textbf{\huge SINH TRẮC HỌC}
    \\
    % (THEO PHƯƠNG PHÁP CÁC NÉT VẼ MÀU - \textcolor{red}{cần thể hiện chi tiết hướng nghiên cứu})

    %Tên đề tài bằng tiếng Anh (nếu có)
    \vspace{.5cm}
    \textit{\textbf{\Large CHỦ ĐỀ: NHẬN DIỆN DÁNG ĐI}}
\end{center}

\vspace{.5cm}

\Large
\section{THÔNG TIN CHUNG}
\begin{itemize}[label = {}]

    \item \textbf{Giảng viên hướng dẫn:}
          %Thể hiện dạng: <Chức danh> <Họ và tên> (<Đơn vị công tác>)
          \begin{itemize}
              \item PGS. TS. Lê Hoàng Thái (Khoa Công nghệ thông tin)
              \item Thầy Dương Thái Bảo (Khoa Công nghệ thông tin) - \textcolor{red}{Tìm kiếm học vị của Thầy (ThS? TS?,...) và bổ sung trước khi nộp}
          \end{itemize}{}

    \item \textbf{Nhóm sinh viên thực hiện:}

          %Thể hiện dạng: <Họ và tên sinh viên> (MSSV: )
          \begin{enumerate}

              \item Phạm Thái Huy (MSSV: 21120081)
              \item Nguyễn Đức Mạnh (MSSV: 22120204)
              \item Lê Quang Vĩnh Quyền (MSSV: 22120307)

          \end{enumerate}

          %Chọn loại thích hợp
          % \item \textbf{Loại đề tài:} Nghiên cứu

          % \item \textbf{Thời gian thực hiện:} Từ \textit{01/2025} đến \textit{07/2025}

\end{itemize}

\pagebreak

\section{NỘI DUNG BÁO CÁO}

\subsection{Tổng hợp nội dung chương sách được chọn}

\textbf{Quyền phụ trách phần 1,4,5,P3D}\\

\textbf{Mạnh phụ trách phần 2,3}

Tổng hợp đầy đủ nội dung, với tổ chức thư mục \textbf{BẮT BUỘC} giống hoàn toàn của chương.
\textbf{Lưu ý:}
\begin{itemize}
    \item Đề cập đầy đủ thông tin cốt lõi. Bỏ qua các \textit{kể chuyện dài dòng} (tóm tắt lại nếu cần).
    \item Các biểu thức toán phải được viết bằng môi trường toán của \LaTeX.
    \item Đề cập đầy đủ các số liệu.
    \item Chèn hình ảnh nếu cần thiết.
    \item \textcolor{red}{Không sao chép nội dung trong sách}.
\end{itemize}
\subsubsection{Giới thiệu}

Với mục tiêu định danh con người từ khoảng cách xa và trong môi trường không bị ràng buộc, nhận diện dáng đi tập trung phân tích và xác định danh tính của cá nhân dựa trên các kiểu đi bộ đặc trưng hoặc cách thức vận động của họ qua các chuỗi hình ảnh được thu thập từ camera. 
Khác với các đặc điểm sinh trắc học vật lý truyền thống như dấu vân tay, khuôn mặt hay mống mắt, dáng đi được xếp vào nhóm sinh trắc học hành vi. Đặc điểm này được hình thành từ cấu trúc cơ xương độc nhất và thói quen vận động của mỗi người, tạo nên một dấu ấn riêng biệt có khả năng phân biệt cao. 
Ngay từ những năm 1970, các nghiên cứu tâm lý học nền tảng của Cutting và Kozlowski đã chứng minh rằng con người có khả năng nhận diện người quen chỉ thông qua sự chuyển động của các điểm sáng mô phỏng dáng đi mà không cần bất kỳ thông tin chi tiết nào về ngoại hình hay khuôn mặt.\\

Một trong những lợi thế quan trọng nhất giúp nhận diện dáng đi trở nên nổi bật là khả năng hoạt động hiệu quả trong các điều kiện khó khăn. Trong khi nhận diện khuôn mặt hay mống mắt yêu cầu đối tượng phải hợp tác, đứng gần thiết bị thu nhận và cần hình ảnh độ phân giải cao, nhận diện dáng 
đi có thể thực hiện từ khoảng cách xa, có thể lên tới hàng trăm mét và chấp nhận dữ liệu có độ phân giải thấp. Đặc biệt, đây là một phương thức nhận diện không xâm lấn, nghĩa là hệ thống có thể thu thập dữ liệu thụ động qua camera giám sát mà không cần sự tương tác trực tiếp hay sự chủ động hợp tác 
của đối tượng. Hơn nữa, do dáng đi là một hành vi vô thức bắt nguồn từ cơ chế sinh học, do đó việc ngụy trang hay giả mạo dáng đi trong thời gian dài là vô cùng khó khăn đối với các đối tượng muốn che giấu danh tính.\\ 

\begin{figure}[h]
    \centering

    \includegraphics[width=0.5\linewidth]{images/gait_recognition.jpg} 
    
    \caption{Nhận diện dáng đi.}
    
    \label{fig:gait_modalities}
\end{figure}

Nhìn chung, việc ứng dụng hệ thống nhận diện dáng đi hiện thực phải đối mặt với rất nhiều thử thách lớn đến từ các yếu tố ngoại cảnh. Độ chính xác của hệ thống thường sẽ rất dễ bị ảnh hưởng nghiêm trọng bởi sự thay đổi của góc nhìn của camera, đây được xem là yếu tố gây nhiễu lớn nhất làm 
thay đổi hình dạng hình học của đối tượng trên khung hình. Bên cạnh đó, các điều kiện về trang phục như áo khoác dày che khuất cơ thể, hoặc trạng thái mang vác vật dụng như ba lô, túi xách cũng làm thay đổi trọng tâm và biên độ dao động của dáng đi. Các yếu tố môi trường khác như bề mặt đường đi, 
điều kiện ánh sáng hay sự che khuất bởi vật cản cũng góp phần làm giảm hiệu suất nhận diện khi áp dụng vào các tình huống đời sống.\\
\subsubsection{Vấn đề}
\subsubsection{Phương pháp}
\subsubsection{Thảo luận và Hướng nghiên cứu tiếp theo}

\textbf{Hình dáng và động lực học của dáng đi}

Trong nghiên cứu về dáng đi, sự tranh luận giữa vai trò của hình dáng và động lực học luôn là tiêu điểm. Các nghiên cứu thực nghiệm ban đầu chỉ ra rằng con người có khả năng nhận diện danh tính dựa trên các đặc điểm động lực học ngay cả khi hình dáng bị che khuất, tuy nhiên các phương pháp phân 
tích dựa trên hình dáng hình bóng lại mang lại hiệu quả vượt trội. Nhiều thuật toán tiên tiến gần đây đã chứng minh rằng việc tập trung vào hình dáng của từng giai đoạn trong chu kỳ bước đi giúp hệ thống đạt được độ chính xác cao hơn, đặc biệt là khi phải đối mặt với các biến số khó như thay đổi 
bề mặt đi bộ. Mặc dù vậy, động lực học vẫn đóng vai trò không thể thiếu vì nó chứa đựng các thông tin về tốc độ và sự chuyển tiếp giữa các pha vận động vốn mang tính đặc trưng cho từng cá nhân. Tuy nhiên, các nghiên cứu mới gần đây chỉ ra rằng việc phụ thuộc quá nhiều vào hình dáng sẽ khiến hệ 
thống dễ bị sai lệch khi đối tượng thay đổi trang phục. Do đó, xu hướng hiện tại là phát triển các mô hình học biểu diễn không gian - thời gian phân cấp, cho phép tách biệt các đặc trưng chuyển động cốt lõi ra khỏi các đặc điểm hình dáng bề ngoài dễ thay đổi, từ đó tận dụng sức mạnh của cả hai yếu tố này.

\begin{figure}[h]
    \centering
    \includegraphics[width=1\linewidth]{images/gait_cycle.jpg} 
    
    \caption{Minh họa chu kỳ dáng đi được chia thành bốn giai đoạn: (i) tựa chân phải; (ii) lăng chân trái; (iii) tựa chân trái; và (iv) lăng chân phải, tương ứng với các trạng thái từ (a) đến (e). Khoảng thời gian cả hai chân cùng tiếp xúc mặt sàn được gọi là giai đoạn hỗ trợ kép.}
    \label{fig:gait_cycle}
\end{figure}

\textbf{Chất lượng hình bóng và nhận dạng dáng đi}

Chất lượng của các hình bóng phụ thuộc vào khả năng phân biệt giữa nền và đối đượng. Trong môi trường ngoài trời, các yếu tố nhiễu như bóng đổ, thay đổi ánh sáng và sự chuyển động của hậu cảnh khiến việc tách hình bóng trở nên cực kỳ khó khăn. Tuy nhiên, một phát hiện chỉ ra rằng sự sụt giảm 
hiệu suất khi thay đổi bề mặt hoặc thời gian không hoàn toàn do lỗi xử lý hình bóng ở cấp độ thấp gây ra. Ngay cả khi sử dụng các hình bóng đã được tiền xử lý thủ công, kết quả vẫn cho thấy sự suy giảm đáng kể, nghĩa là bản thân dáng đi của con người đã có những thay đổi cơ bản khi điều kiện 
môi trường thay đổi. Điều này cho thấy các công trình tương tai thay vì tìm kiếm các phương pháp tốt hơn để phát hiện hình bóng nhằm cải thiện nhận dạng, việc nghiên cứu và tách biệt các thành phần của dáng đi không thay đổi theo giày dép, bề mặt hoặc thời gian sẽ hiệu quả hơn.

\textbf{Các biến số}

Thách thức của nhận diện dáng đi nằm ở việc duy trì độ ổn định trước các tác động của các yếu tố ngoại cảnh. Trong khi các yếu tố như loại giày dép hay việc mang theo túi xách có tác động tương đối nhỏ, thì sự thay đổi về bề mặt đi bộ và khoảng cách thời gian giữa các lần thu thập dữ 
liệu lại gây ra những ảnh hưởng tiêu cực nghiêm trọng. Đặc biệt, nhận dạng dáng đi theo thời gian là một bài toán khó do sự thay đổi về trang phục theo mùa và những biến đổi tự nhiên trong cơ thể người theo thời gian. Ngoài ra, việc di chuyển trong các môi trường thực tế với các góc nhìn 
camera đa dạng, vật cản che khuất và độ phân giải thấp vẫn là những trở ngại cần được giải quyết. Do đó các nghiên cứu trong tương lai cần tập trung vào việc mô hình hóa các thành phần dáng đi không thay đổi hoặc tìm cách dự đoán sự biến đổi của dáng đi khi chuyển từ bề mặt này sang bề mặt khác.

\textbf{Các bộ dữ liệu trong tương lai}

Sự phát triển của nhận dạng dáng đi phụ thuộc rất lớn vào quy mô và độ đa dạng của các bộ dữ liệu. Các chuyên gia nhận định rằng cần có những bộ dữ liệu khổng lồ với quy mô lên tới hàng nghìn đối tượng để hỗ trợ việc hiểu sâu hơn về sự biến thiên của dáng đi trong điều kiện ngoài trời và theo thời 
gian. Các bộ dữ liệu chuẩn như CASIA-B hay OU-MVLP đang dần trở nên bão hòa khi các mô hình Học sâu đã đạt độ chính xác rất cao trên đó. Hiện nay, các bộ dữ liệu như GREW hay Gait3D đã bắt đầu chuyển hướng từ môi trường phòng thí nghiệm sang môi trường thực tế với hàng chục nghìn danh tính và hàng 
triệu chuỗi hình ảnh. Một hướng đi mới đầy triển vọng là sử dụng dữ liệu tổng hợp được tạo ra từ các mô hình cơ thể người ảo, giúp giải quyết vấn đề thiếu hụt dữ liệu được dán nhãn và giảm bớt các rào cản về chi phí thu thập dữ liệu thực tế. Đồng thời, việc khai thác dữ liệu video không nhãn trên 
quy mô lớn thông qua các phương pháp học tự giám sát đang trở thành một lĩnh vực nghiên cứu đầy tiềm năng.

\textbf{Kết hợp khuôn mặt và dáng đi}

Mặc dù dáng đi có ưu thế ở khoảng cách xa, thì việc kết hợp giữa nhận dạng dáng đi và khuôn mặt là một giải pháp tối ưu để nâng cao độ tin cậy của hệ thống. Dáng đi có thể được sử dụng khi đối tượng ở khoảng cách xa, trong khi khuôn mặt sẽ phát huy tác dụng khi đối tượng tiến lại gần camera hơn. 
Việc kết hợp đa phương thức dáng đi và mặt người mang lại hiệu suất vượt trội so với việc chỉ sử dụng một loại sinh trắc học đơn lẻ, đồng thời giúp hệ thống bền bỉ hơn trước các nhiễu động của từng loại dữ liệu. Trong tương lai, việc tích hợp này không chỉ dừng lại ở khuôn mặt mà còn có thể mở 
rộng sang các phương thức khác như thông tin về chiều cao, kích thước các bộ phận cơ thể hoặc dữ liệu từ nhiều góc nhìn camera cùng lúc để tạo ra một hồ sơ định danh toàn diện và chính xác hơn

\begin{figure}[h]
    \centering

    \includegraphics[width=0.8\linewidth]{images/face_samples.jpg} 
    
    \caption{Các mẫu khuôn mặt dưới nhiều điều kiện khác nhau: (a) và (b) là ảnh tập mẫu trong điều kiện ánh sáng khác nhau; (c) và (d) là ảnh tập kiểm tra chụp ngoài trời ở khoảng cách xa và gần.}
    
    \label{fig:face_samples}
\end{figure}

\textbf{Quyền riêng tư và bảo mật sinh trắc học}

Một khía cạnh mới đang ngày càng được chú trọng là tính bảo mật và quyền riêng tư của dữ liệu dáng đi. Do dáng đi có thể được thu thập từ xa mà không cần sự hợp tác hay nhận biết của đối tượng, công nghệ này làm dấy lên những lo ngại nghiêm trọng về quyền riêng tư cá nhân và sự giám sát đại 
chúng. Bên cạnh đó, vấn đề an ninh của chính hệ thống AI cũng đáng báo động với sự xuất hiện của các cuộc tấn công đối kháng, nơi kẻ tấn công có thể thay đổi một chút dáng đi hoặc thêm nhiễu vào video để đánh lừa hệ thống. Các luật lệ như quy định chung về Bảo vệ Dữ liệu Châu Âu (GDPR) đã 
đặt ra những hạn chế chặt chẽ đối với việc sử dụng dữ liệu sinh trắc học, thúc đẩy cộng đồng nghiên cứu phải tìm kiếm các giải pháp bảo vệ quyền riêng tư. Các hướng nghiên cứu mới bao gồm việc phát triển các phương pháp ẩn danh hóa, mã hóa video dáng đi sao cho hệ thống nhận dạng vẫn hoạt 
động nhưng danh tính con người không thể bị quan sát bằng mắt thường, cũng như tăng cường khả năng chống lại các cuộc tấn công giả mạo hoặc tấn công đối nghịch nhằm đánh lừa hệ thống nhận dạng.

\subsubsection{Kết luận}

Nhìn lại toàn bộ quá trình phát triển, nhận dạng dáng đi đã khẳng định vị thế là một trong những công nghệ sinh trắc học tiềm năng nhất, với điểm mạnh về khả năng định danh tầm xa và không xâm lấn mà các phương pháp truyền thống như khuôn mặt hay vân tay không thể thay thế. 
Sự chuyển dịch mạnh mẽ từ các kỹ thuật thị giác máy tính cổ điển sang các kiến trúc Học sâu tiên tiến đã nâng tầm lĩnh vực này, giúp các hệ thống hiện đại đạt được độ chính xác ấn tượng trên các bộ dữ liệu tiêu chuẩn. Các nghiên cứu đột phá gần đây đã chứng minh rằng việc khai thác sâu các biểu diễn 
không gian - thời gian phân cấp là chìa khóa để tách biệt các đặc trưng vận động cốt lõi khỏi những yếu tố nhiễu loạn của bề mặt, mở ra triển vọng to lớn trong việc giải mã hành vi vận động của con người.

Tuy nhiên, thực tế triển khai đã chỉ ra một khoảng cách đáng kể về tính thực tiễn giữa môi trường phòng thí nghiệm lý tưởng và thế giới thực hỗn loạn. Như một vài phân tích thực nghiệm đã làm rõ hiệu suất của các thuật toán hàng đầu vẫn sụt giảm nghiêm trọng khi đối mặt với sự đa dạng không giới hạn 
của các biến số ngoại cảnh như góc quay camera an ninh phức tạp, sự thay đổi trang phục theo mùa, hay điều kiện ánh sáng và vật che khuất trong môi trường tự nhiên. Điều này khẳng định rằng, mặc dù chúng ta đã giải quyết tốt bài toán so khớp mẫu trong điều kiện kiểm soát, nhưng bài toán nhận dạng ở 
ngoài thực tế vẫn là thách thức lớn cần tiếp tục giải quyết.

Trong tương lai, sự phát triển của nhận dạng dáng đi sẽ không còn đơn thuần là cuộc đua về các chỉ số độ chính xác trên tập dữ liệu cũ, mà sẽ là sự chuyển mình sang các hệ thống thông minh, bền vững và an toàn hơn. Xu hướng tất yếu sẽ là sự kết hợp đa phương thức giữa dữ liệu để vượt qua giới hạn của 
camera quang học, cùng với việc áp dụng các kỹ thuật học không giám sát để tận dụng nguồn dữ liệu khổng lồ chưa gán nhãn. Đồng thời, khi công nghệ này đi sâu vào đời sống, các vấn đề về bảo mật chống giả mạo và bảo tồn quyền riêng tư sẽ trở thành những trụ cột quan trọng ngang hàng với hiệu năng kỹ thuật, 
đảm bảo rằng nhận dạng dáng đi không chỉ là một công cụ giám sát hiệu quả mà còn là một công nghệ có trách nhiệm và đáng tin cậy.

\subsection{Phương pháp trình bày}

\textbf{Người phụ trách:} Huy.

\begin{itemize}
    \item Trình bày chi tiết về phương pháp gốc đã chọn. (đặt vấn đề, đề xuất phương pháp, tiến hành thực nghiệm, phân tích kết quả, bàn luận, tổng kết)
    \item Phân tích của nhóm về hạn chế tiềm ẩn của phương pháp.
\end{itemize}

\subsection{Hướng nghiên cứu và thực nghiệm}

Hướng nghiên cứu: Dựa trên phân tích về hạn chế tiềm ẩn, nhóm đều xuất các giải pháp thay thế, tái thực nghiệm, so sánh với phương pháp gốc, phân tích và bàn luận.
\newpage

\subsubsection{Thay thế kỹ thuật ARME thành P3D}
\textbf{Cơ sơ sở lý thuyết}

Xét một thao tác tích chập trên một vùng cơ thể thứ $j$ tại cấp độ $l$. Giả sử đầu vào là tensor $X \in \mathbb{R}^{C_{in} \times T \times H \times W}$.
Bộ lọc có kích thước không gian $k \times k$  và thời gian $d$.

Đối với kĩ thuật ARME, thì ARME sử dụng tích chập 3D tiêu chuẩn để trích xuất đặc trưng không gian và thời gian đồng thời.
Cụ thể là dùng một kernel 3D kích thước $d \times k \times k$ trượt trên cả ba chiều (thời gian, chiều cao, chiều rộng).\\
Với mỗi vùng $j$, đầu ra $Y_j$ được tính bằng:$$Y_j = f_j(X_j) = W_{3D} \ast X_j + b$$Trong đó $W_{3D} \in \mathbb{R}^{C_{out} \times C_{in} \times d \times k \times k}$.



Đối với kĩ thuật P3D, P3D tách một kernel 3D kích thước $d \times k \times k$ thành hai kernel riêng biệt: một kernel không gian ($1 \times k \times k$) và một kernel thời gian ($d \times 1 \times 1$).
Theo cơ chế như sau:
\begin{itemize}
    \item Kernel không gian ($S$): Sử dụng bộ lọc kích thước $1 \times k \times k$, tương đương với 2D CNN để học các đặc trưng hình ảnh.
    \item Kernel thời gian ($T$): Sử dụng các bộ lọc $d \times 1 \times 1$ để xây dựng các kết nối thời gian giữa các bản đồ đặc trưng liền kề.
\end{itemize}

Thay vì tính trực tiếp, ta thực hiện tuần tự:
$$Y_{Spatial} = S(X_j) = W_{S} \ast X_j \quad (\text{kernel } 1 \times k \times k)$$
$$Y_{j} = T(Y_{Spatial}) = W_{T} \ast Y_{Spatial} \quad (\text{kernel } d \times 1 \times 1)$$

Trong đó $W_{S} \in \mathbb{R}^{C_{out} \times C_{in} \times 1 \times k \times k}$ và $W_{T} \in \mathbb{R}^{C_{out} \times C_{out} \times d \times 1 \times 1}$.\\
Trong nghiên cứu này, tôi áp dụng cấu trúc Residual của P3D-A. Cụ thể là thành phần thời gian (T) đi trực tiếp sau thành phần không gian (S) trên cùng một đường dẫn. Đầu ra của không gian là đầu vào của thời gian.

$$x_{t+1} = (I + T \cdot S) \cdot x_t = x_t + T(S(x_t))$$

\begin{figure}[h]
    \centering
    
    \includegraphics[width=0.3\linewidth]{images/p3d_a.jpg} 
    
    \caption{Kiến trúc của khối P3D-A: Các bộ lọc không gian (S) và thời gian (T) được sắp xếp nối tiếp.}
    
    \label{fig:p3d_a}
\end{figure}

\textbf{Ưu nhược điểm và độ phức tạp tính toán}

Về mặt tính toán, module ARME trong HSTL sử dụng tích chập 3D tiêu chuẩn ($3 \times 3 \times 3$) nên tốn kém tài nguyên với số lượng tham số tỷ lệ thuận với $3 \times 3 \times 3$ = 27. Trong khi đó, kỹ thuật P3D tách kernel này thành hai phần riêng biệt: 
không gian ($1 \times 3 \times 3$) và thời gian ($3 \times 1 \times 1$), giúp giảm số lượng tham số xuống chỉ còn tỷ lệ với $1 \times 3 \times 3$ + $3 \times 1 \times 1$ = 12. Như vậy, việc chuyển sang P3D giúp giảm khối lượng tính toán và tham số khoảng 2.25 lần. Sự tối ưu này 
cực kỳ quan trọng đối với kiến trúc chia vùng của HSTL, cho phép bạn xây dựng mô hình sâu hơn hoặc xử lý dữ liệu lớn hơn mà không bị quá tải bộ nhớ.

Mặc dù P3D đã giảm đáng kể số lượng tham số và chi phí tính toán, tuy nhiên việc tách biệt không gian và thời gian có thể làm giảm khả năng học các mối tương quan chặt chẽ tức thời giữa hai miền này so với ARME.

\textbf{Mức độ cải thiện kỳ vọng}

Việc thay thế ARME bằng P3D được kỳ vọng sẽ mang lại các lợi ích sau:
\begin{itemize}
    \item Giảm đáng kể chi phí tính toán và bộ nhớ, cho phép mô hình xử lý các chuỗi video dài hơn hoặc nhiều vùng cơ thể hơn mà không bị quá tải.
    \item Vì P3D sẽ có khả năng khái quát hóa cực tốt trên nhiều tác vụ video và bộ dữ liệu khác nhau như Sports-1M, UCF101. Khi đưa vào HSTL, nó có thể giúp mô hình hoạt động ổn định hơn trên các tập dữ liệu ngoài thực tế như GREW hoặc Gait3D. 
    \item Trong nhận diện dáng người, việc sử dụng tích chập P3D trong tập CASIA-B hy vọng sẽ cải thiện độ chính xác như trong DeepGaitV2-P3D đã cải thiện độ chính xác tới 9.1\% so với phiên bản 2D thuần túy trên một số bộ dữ liệu.
    
\end{itemize}


Lưu ý:
\begin{itemize}
    \item Phân tích chi tiết giữa kĩ thuật gốc và kĩ thuật đề xuất (ưu nhược điểm, độ phức tạp, biểu thức toán, mức độ cải thiện kỳ vọng đóng góp vào hiệu suất của toàn mô hình).
    \item \textcolor{red}{Không chèn mã nguồn vào báo cáo}.
\end{itemize}
\begin{table}[H]
\centering
\begin{tabular}{|l|l|}
\hline
\textbf{Hướng đề xuất} & \textbf{Người phụ trách} \\ \hline
Thay \textbf{Conv3D} bằng \textbf{P3D} & Mạnh, Quyền \\ \hline
Thay \textbf{Triplet Loss} bằng \textbf{Circle Loss} & Huy \\ \hline
Bổ sung kĩ thuật \textbf{TSM} vào module chứa \textbf{Conv3D} & Huy \\ \hline
\end{tabular}
\end{table}

\section{LỜI KẾT}

% Cuối lời, chúng em chân thành cảm ơn Thầy vì đã cho chúng em một cơ hội để trình bày cũng như lắng nghe những nhận xét quý báu, nhờ đó chúng em có thể rút kinh nghiệm và cải thiện hơn cho phần trình bày trong buổi bảo vệ khóa luận sắp tới, cũng như quá trình học tập, nghiên cứu và làm việc sau này. Vì đây cũng là lần đầu tiên chúng em thực hiện một nghiên cứu mang tính học thuật cao, do đó không thể nào tránh khỏi những sai sót trong quá trình thực hiện, chúng em rất mong Thầy có thể thông cảm. Đồng thời chúng em cũng rất mong nhận được thêm các lời nhận xét và góp ý từ thầy để chúng em có thể tiếp thu, cải thiện và ngày càng phát triển.

% Cuối lời nhóm chúng em xin chân thành cảm ơn Thầy, chúc Thầy và gia đình nhiều sức khỏe.

% Trân trọng.

    

\textcolor{red}{\textbf{DEADLINE: 22h - 25/12/2025}}

\section{TÀI LIỆU THAM KHẢO}
\bibliographystyle{plain} 
\bibliography{tailieu}

\end{document}


% Mục lục, danh sách hình, danh sách bảng
\addcontentsline{toc}{chapter}{Mục lục}

\tableofcontents

%\listoffigures
\cleardoublepage
% \phantomsection
\addcontentsline{toc}{chapter}{\listfigurename}
\listoffigures

\cleardoublepage
% \phantomsection
\addcontentsline{toc}{chapter}{\listtablename}
\listoftables
%\listoftables

\addcontentsline{toc}{chapter}{Tóm tắt}
\chapter*{Tóm tắt}
\label{summary}

\textcolor{red}{Phần quy định chung khi trình bày báo cáo:}
\begin{itemize}
    \item Tất cả biểu thức toán phải được đánh số.
    \item Tất cả hình vẽ, bảng biểu phải có chú thích, và đánh dấu để tham chiếu trong mục lục.
    \item Hạn chế tối đa việc dùng tiếng anh,kể cả các thuật ngữ chuyên môn, có gắng dịch sang tiếng việt với từ ngữ dễ hiểu. Trong trường hợp quá khó để phiên dịch, được phép dùng tiếng anh, nhưng phải bổ sung ở bảng \ref{tab:vietanh} trong phần \textbf{Phụ lục}.
    \item Các tài liệu đính kèm phải theo một chuẩn duy nhất, không tham khảo loạn xạ. Có nhiều cách để lấy định dạng chuẩn, cách đơn giản nhất là tìm bài báo đó trên trang \textbf{\href{https://arxiv.org/}{arxiv}}, tìm dòng \textbf{export BibTeX citation}, sao chép định dạng đó vào file \textbf{references.bib}.
\end{itemize}

\textcolor{blue}{Viết theo mẫu bên dưới\\}
Tô màu ảnh độ xám là quá trình ước lượng các giá trị màu RGB cho từng điểm trong ảnh độ xám, nhằm chuyển đổi chúng thành ảnh màu. Tô màu ảnh độ xám được chia thành hai nhóm lớn: tô màu tự động và tô màu bán tự động. Tuy nhiên, hầu hết các mô hình hiện tại chỉ tập trung vào một phương thức duy nhất, điều này hạn chế tính năng mà người dùng mong muốn.

Trong nghiên cứu này, nhóm giới thiệu một mô hình tô màu ảnh độ xám kết hợp cả hai phương thức tự động và bán tự động (dựa trên lời nhắc văn bản) được phát triển dựa trên kiến trúc mô hình ControlNet. ControlNet là một kiến trúc mạng nơ-ron được thiết kế nhằm tích hợp các điều kiện điều khiển không gian vào các mô hình tạo sinh ảnh từ văn bản đã được huấn luyện trước.

Nhóm đã tiến hành huấn luyện mô hình trên bốn tập dữ liệu cho các mục đích khác nhau. Sau đó nhóm tiến hành thử nghiệm mô hình tô màu chính trên ba tập dữ liệu đánh giá khác nhau. Kết quả đánh giá cho thấy, mô hình của nhóm đã vượt qua hầu hết các phương pháp tô màu tiên tiến trước đó trên phương thức tô màu tự động về chỉ số Colorfulness. Cuối cùng, nhóm xây dựng thêm một giao diện đơn giản để người dùng có thể sử dụng và ứng dụng các mô hình vào các công việc có liên quan đến tô màu ảnh độ xám.



\clearpage

\pagenumbering{arabic} % Đánh số 1, 2, 3, ...

% Các chương nội dung
\chapter{Giới thiệu}
\label{Chapter1}
\textcolor{blue}{Mục này được viết dựa trên các bài khảo sát (survey) để giới thiệu tổng quan về chủ đề lớn \textcolor{red}{Nhận dạng dáng đi} được nghiên cứu. Cách tổ chức nội dung bên dưới có thể tham khảo, hoặc chỉnh sửa tùy cho phù hợp ngữ cảnh.}
\section{Bối cảnh chung}

Với sự phát triển của trí tuệ nhân tạo, các ứng dụng xử lý ảnh ngày càng được cải thiện, đặc biệt trong các tác vụ dịch ảnh sang ảnh như tô màu ảnh độ xám. Nhu cầu phục chế màu cho các bức ảnh trắng đen ngày càng gia tăng, đặc biệt trong các ứng dụng liên quan đến lịch sử và xã hội.

Về mặt thực tiễn, nhu cầu tô màu ảnh độ xám đến từ nhiều lĩnh vực như phục dựng ảnh lịch sử, tô màu ảnh phác họa trong thiết kế. Về mặt nghiên cứu, một số vấn đề trong chủ đề tô màu ảnh độ xám vẫn chưa được giải quyết triệt để, bao gồm phương thức tô màu tự động và bán tự động, vấn đề về dữ liệu huấn luyện, và chi phí tính toán.

\section{Động lực}

Kỳ vọng của nghiên cứu này là tạo ra một mô hình tô màu có thể đáp ứng các ứng dụng như phục hồi ảnh trắng đen lịch sử. Sản phẩm cuối cùng sẽ cho phép người dùng có thể tô màu theo cả hai phương pháp, tô màu tự động hoặc bán tự động dựa trên nhu cầu.

Bài toán tô màu là một bài toán tạo sinh ảnh không đơn trị, tức là từ một ảnh độ xám có thể có nhiều cách tô màu hợp lý khác nhau. Do đó, một mô hình hiệu quả không chỉ cần tái tạo ảnh màu đẹp mắt mà còn cần hiểu được ngữ cảnh và nội dung ngữ nghĩa trong ảnh.

\section{Phát biểu bài toán}

Bài toán tô màu ảnh độ xám là việc \textbf{ước tính các giá trị màu của các điểm ảnh trong một ảnh độ xám}. Hệ thống tô màu một ảnh độ xám sẽ gồm các thành phần:

\begin{itemize}
    \item \textbf{Đầu vào:} $I_g$ là ma trận đại diện cho \textbf{ảnh độ xám} cần được tô và $c$ là điều kiện điều khiển việc tô màu (nếu có).
    \item \textbf{Đầu ra:} $I_{rgb}$ là ảnh màu trong không gian RGB do hệ thống trả về.
\end{itemize}

Phương trình dự đoán ảnh màu có thể viết như sau:

\begin{equation} \label{eqn:general}
    I_{rgb} = \Phi(I_g).
\end{equation}

Nếu quá trình tô màu được điều khiển bởi điều kiện $c$, phương trình \ref{eqn:general} có thể được viết lại thành:

\begin{equation} \label{eqn:generalcon}
    I_{rgb} = \Phi(I_g, c).
\end{equation}

Với $\Phi(.)$ là hàm số được xây dựng để tạo ra ảnh màu từ ảnh độ xám đầu vào một cách chân thật và phù hợp với điều kiện điều khiển.

\section{Thách thức của bài toán}

Thách thức đầu tiên là vấn đề rất khó để có thể kiểm tra được kết quả của quá trình tô màu bằng phương pháp định lượng. Với một bức ảnh độ xám đầu vào, ta gần như có vô số cách tô màu, và kết quả đó có phù hợp hay không chỉ có thể được đánh giá qua cảm nhận của người sử dụng.

Một thách thức khác là việc nhiều bức ảnh cùng chụp 1 đối tượng nhưng có thể bị ảnh hưởng bởi nhiều yếu tố khác nhau như góc chụp, độ chói, độ sáng. Ngoài ra, chúng ta cũng rất khó có thể xác định màu cho một số đối tượng có màu tùy biến như quần áo, phương tiện đi lại, đồ nội thất.

\section{Đóng góp}

Trong khóa luận này, mục tiêu là xây dựng một mô hình tô màu ảnh độ xám đa phương thức dựa trên mô hình khuếch tán. Đóng góp bao gồm:

\textbf{Về mặt lý thuyết:}

\begin{itemize}
    \item Xây dựng một phương pháp tô màu ảnh độ xám đa phương thức dựa trên mô hình khuếch tán có kết hợp điều kiện.
    \item Kết hợp bộ giải mã biến dạng để giải quyết vấn đề tràn màu và tô màu sai.
    \item Kết hợp mô hình ngôn ngữ ảnh BLIP vào quá trình tạo sinh dữ liệu văn bản.
\end{itemize}

\textbf{Về mặt thực tiễn:}

\begin{itemize}
    \item Thu thập, xử lý và cung cấp các tập dữ liệu dùng cho quá trình huấn luyện.
    \item Tạo ra một mô hình tô màu ảnh độ xám chung cho nhiều miền dữ liệu.
    \item Cung cấp một giao diện đơn giản, dễ sử dụng và có khả năng tùy chỉnh cao.
\end{itemize}

\section{Bố cục của khóa luận}

Báo cáo khóa luận bao gồm năm chương: Chương \ref{Chapter1} trình bày bối cảnh, động lực nghiên cứu và phát biểu bài toán; Chương \ref{Chapter2} trình bày các công trình nghiên cứu liên quan; Chương \ref{Chapter3} trình bày chi tiết phương pháp đề xuất; Chương \ref{Chapter4} trình bày kết quả thực nghiệm; Chương \ref{Chapter5} trình bày kết luận và hướng phát triển.


\chapter{Các công trình liên quan}
\label{Chapter2}
\textcolor{blue}{Mục này đi sâu hơn vào các nghiên cứu/phương pháp cho chủ đề \textcolor{red}{Nhận dạng dáng đi}. Gợi ý:}
\begin{itemize}
    \item \textcolor{blue}{Phần lớn nội dung mục này dựa trên các bài tổng hợp/khảo sát.}
    \item Liệt kê tất cả các phương pháp đến thời điểm hiện tại \textcolor{red}{2026}. Một số cách trình bày (chắc chắn được dùng trong các bài tổng hợp, nên tham khảo kĩ):
          \begin{itemize}
              \item Vẽ lược đồ thời gian công bố của các nghiên cứu.
              \item Vẽ sơ đồ chia các nghiên cứu theo các nhánh tiếp cận chính.
          \end{itemize}
    \item Từng nhánh tiếp cận bài toán sẽ trình bày sâu hơn trong từng mục nhỏ:
          \begin{itemize}
              \item  Liệt kê theo trình tự thời gian
              \item Trình ý tưởng/nguyên lý của từng nghiên cứu được đề cập.
              \item Kết luận bằng điểm mạnh và hạn chế chung của nhóm các phương pháp.
          \end{itemize}
    \item \textcolor{red}{Lưu ý:} Khi trình bày đến nhánh tiếp cận chính (\textcolor{blue}{có chứa nghiên cứu được chọn (HSTL)}), nên trình bày nhiều hơn cả về điểm mạnh (ngầm ý đây là phương pháp được quan tâm) và điểm yếu (ngầm ý đóng góp của nghiên cứu này).
    \item Đối với các nhánh tiếp cận hoặc nghiên cứu vượt trội nghiên cứu được chọn (\textcolor{blue}{HSTL}) về hiệu năng, hãy nói rõ hạn chế lớn của chúng. Quan trọng hơn hết, hãy trung thực về lý do không lựa chọn đào sâu, ví dụ, chi phí (huấn luyện, lưu trữ dữ liệu) lớn (\textcolor{blue}{gần như là điều hiển nhiên khi chi phí lớn sẽ cho hiệu năng lớn}), \textcolor{blue}{năng lực nghiên cứu hạn chế của nhóm}, \dots
    \item Để người đọc dễ tiếp cận, nên chèn thêm các hình vẽ minh họa về dữ liệu của từng nhánh tiếp cận.
    \item \textcolor{red}{Nội dung có sẵn khá lan man, không nên viết theo.}
\end{itemize}

\section{Sự phát triển của tô màu ảnh độ xám}

Tô màu ảnh độ xám đã trải qua nhiều giai đoạn phát triển, đặc biệt với sự xuất hiện của các công nghệ học sâu. Ban đầu, các phương pháp tô màu chủ yếu dựa vào kỹ thuật thủ công như được đề cập tại bài viết của Ethan \cite{mental_floss_colorization}.

\subsection{Các phương pháp truyền thống}

Trước khi học sâu trở thành xu hướng, các phương pháp tô màu ảnh chủ yếu dựa vào các thuật toán xử lý ảnh truyền thống. Tuy nhiên như được trình bày trong nghiên cứu của Levin và các đồng nghiệp \cite{Levin_Lischinski_Weiss_2004}, những phương pháp này thường gặp nhiều khó khăn trong việc tạo ra màu sắc tự nhiên và chân thực.

\subsection{Sự xuất hiện của mạng nơ-ron tích chập}

Sự phát triển của mạng nơ-ron tích chập đã đánh dấu một bước ngoặt quan trọng. Nghiên cứu của Zhang và các đồng nghiệp \cite{Zhang_Isola_Efros_2016} đã giới thiệu mô hình "Colorful Image Colorization", sử dụng mạng nơ-ron tích chập để tự động dự đoán màu sắc. Nghiên cứu của Iizuka \cite{Iizuka_Simo-Serra_Ishikawa_2016} cũng đã đóng góp với việc phát triển một hệ thống tô màu có khả năng xử lý các bức ảnh với độ phức tạp cao.

\subsection{Mạng đối kháng tạo sinh}

Sự phát triển của các mạng tạo sinh đối kháng đã mở ra những khả năng mới. DeOldify \cite{DeOldify} đã áp dụng mạng này để phục hồi và tô màu cho các bức ảnh lịch sử. BigColor \cite{avidan_bigcolor_2022} và ChromaGAN \cite{vitoria_chromagan_2020} đại diện cho hai hướng tiếp cận bổ sung lẫn nhau trong lĩnh vực này.

\subsection{Kỹ thuật khuếch tán}

Gần đây, kỹ thuật khuếch tán đã được áp dụng vào lĩnh vực tạo sinh ảnh. Các mô hình khuếch tán như DALL-E 2 \cite{Ramesh_Dhariwal_Nichol_Chu_Chen_2022} và Stable Diffusion \cite{LDM_Rombach_Blattmann_Lorenz_Esser_Ommer_2022} đã cho thấy khả năng tạo ra hình ảnh chất lượng cao. Nghiên cứu của Ho và các đồng nghiệp \cite{Ho_Jain_Abbeel_2020} đã chỉ ra rằng mô hình khuếch tán có thể tạo ra các bức ảnh với độ chi tiết và chân thực cao.

Palette \cite{Palette} là một mô hình tiên phong sử dụng kỹ thuật khuếch tán cho các tác vụ dịch ảnh sang ảnh, bao gồm cả tô màu ảnh độ xám. Nghiên cứu này sử dụng kiến trúc U-Net cùng với kỹ thuật tự chú ý để học quá trình loại bỏ nhiễu từ ảnh màu bị nhiễu.

\section{Tô màu bán tự động và tô màu tự động}

Các phương pháp tô màu có thể được phân chia thành hai loại chính: tô màu tự động và tô màu bán tự động.

\subsection{Tô màu tự động}

Tô màu tự động là một quy trình hoàn toàn tự động, trong đó các thuật toán học sâu được sử dụng để chuyển đổi hình ảnh đen trắng thành màu mà không cần sự can thiệp của người dùng. Mô hình của Zhang và các đồng nghiệp \cite{Zhang_Isola_Efros_2016} là một nghiên cứu nổi bật trong lĩnh vực này. Các mô hình dựa trên mạng tạo sinh đối kháng như DeOldify \cite{DeOldify} hay ChromaGAN \cite{vitoria_chromagan_2020} cũng đã được phát triển.

\subsection{Tô màu bán tự động}

Tô màu bán tự động cho phép người dùng can thiệp vào quá trình tô màu thông qua các điều kiện đầu vào khác nhau. Phương pháp này được chia thành ba loại chính:

\textbf{Tô màu dựa trên nét vẽ:} Phương pháp này cho phép người dùng chỉ định màu cho các khu vực cụ thể bằng các nét vẽ. Các mô hình như Icolorit \cite{Yun_Lee_Park_Choo_2022}, UniColor \cite{huang_unicolor_2022} cho phép sử dụng gợi ý nét vẽ trong tô màu.

\textbf{Tô màu dựa trên ảnh mẫu:} Phương pháp này sử dụng một hình ảnh mẫu để điều chỉnh màu sắc cho ảnh độ xám, như trong nghiên cứu của He và các đồng nghiệp \cite{He_Chen_Liao_Sander_Yuan_2018}.

\textbf{Tô màu dựa trên lời nhắc văn bản:} Phương pháp này sử dụng lời nhắc bằng văn bản để hỗ trợ quá trình tô màu. Nghiên cứu của Weng và các đồng nghiệp \cite{lcode_Weng_Wu_Chang_Tang_Li_Shi_2022} đã tách biệt hai thông tin về đối tượng và màu trong câu hướng dẫn.

UniColor \cite{huang_unicolor_2022} là một mô hình đa phương thức đã kết hợp tất cả ba loại điều kiện bằng cách biểu diễn chúng dưới dạng các điểm gợi ý.

\section{Mô hình khuếch tán trong không gian tiềm ẩn}

Mô hình khuếch tán trong không gian tiềm ẩn được nghiên cứu bởi Rombach và các đồng nghiệp \cite{LDM_Rombach_Blattmann_Lorenz_Esser_Ommer_2022} là một cải tiến của các mô hình khuếch tán truyền thống, trong đó quá trình khuếch tán được thực hiện trong không gian tiềm ẩn thay vì không gian điểm ảnh thông thường. Bằng cách này, mô hình có thể giảm thiểu chi phí tính toán và tăng cường khả năng sinh ảnh.

\section{Kết hợp điều kiện không gian}

ControlNet \cite{ControlNet} là một phương pháp tiên tiến cho phép tích hợp các điều kiện vào trong quá trình sinh ảnh của các mô hình tạo sinh ảnh dựa trên văn bản lớn. Phương pháp này cho phép người dùng cung cấp các đầu vào bổ sung, như các bản đồ ngữ nghĩa, ảnh biên cạnh, để kiểm soát quá trình sinh ảnh một cách ổn định hơn. ControlNet \cite{ControlNet} hoạt động bằng cách thêm một mạng con vào cấu trúc của mô hình tạo sinh gốc.


\chapter{Phương pháp đề xuất}
\label{Chapter3}
\textcolor{blue}{Mục này có cách viết đơn giản nhất là sẽ trình bày lại gần như đầy đủ các nội dung của phương pháp được chọn, nhưng những phần hạn chế sẽ được thay thế bằng các cản tiến do nhóm đề xuất. Cụ thể:}
\begin{itemize}
	\item Vẽ lại sơ đồ phương pháp đề xuất với các thành phần cải tiến được đề xuất.
	\item Đối với cải tiến khối \textbf{Conv3D}, báo cáo này không trình bày lại, mà trình bày thẳng về khối \textbf{Pseudo 3D}. Bên cạnh đó, bổ sung các so sánh lý thuyết giữa kỹ thuật đề xuất so với kĩ thuật gốc ( khác biệt về nguyên lý, độ phức tạp thuật toán, chi phí huấn luyện, tổng tham số của mô hình lớn,\dots)
	\item Tương tự với \textbf{Circle Loss}.
	\item \textcolor{red}{Tất cả các hình vẽ về các kỹ thuật hay phương pháp đề xuất, phải được chèn vào.}
	\item \textcolor{red}{Nội dung có sẵn khá lan man, không nên viết theo.}
\end{itemize}
\section{Tổng quan về mô hình}

Hình \ref{fig:main_model} thể hiện quy trình huấn luyện tổng quát của mô hình, quy trình này bao gồm 2 giai đoạn:

\begin{figure}[htp]
	\centering
	\includegraphics[width=1\linewidth]{figures/main_model.png}
	\caption{Tổng quan về mô hình tô màu ảnh độ xám dựa trên mô hình tổng hợp ảnh có điều kiện sử dụng quy trình khuếch tán.}
	\label{fig:main_model}
\end{figure}

\begin{itemize}
	\item \textbf{Giai đoạn 1:} Thiết lập tín hiệu điều khiển thông qua câu hướng dẫn và ảnh độ xám đầu vào. Mô hình sử dụng bộ mã hóa văn bản $\mathcal{E}_p$ của CLIP \cite{Radford_CLIP_2021} để mã hóa các lời nhắc văn bản được tạo ra bởi mô hình BLIP \cite{blip}. Mô hình tích hợp một mạng mã hóa điều kiện $\mathcal{E}_c$ để trích xuất đặc trưng từ ảnh điều kiện trước khi đưa vào ControlNet \cite{ControlNet}.

	\item \textbf{Giai đoạn 2:} Quá trình khuếch tán có kết hợp điều kiện không gian. Ảnh màu mục tiêu được mã hóa vào không gian tiềm ẩn bằng bộ mã hóa $\mathcal{E}$ và được thêm các nhiễu Gauss qua $T$ bước, sau đó được đưa vào cả 2 khối ControlNet \cite{ControlNet} và Stable Diffusion \cite{LDM_Rombach_Blattmann_Lorenz_Esser_Ommer_2022} cho quá trình khử nhiễu có điều kiện.
\end{itemize}

\section{Thiết lập tín hiệu điều khiển}

\subsection{Bộ mã hóa văn bản của CLIP}

CLIP là một mô hình ngôn ngữ - thị giác được nghiên cứu bởi Radford và các đồng nghiệp \cite{Radford_CLIP_2021}, được huấn luyện trên hàng triệu cặp dữ liệu hình ảnh và văn bản. Ý tưởng chính của CLIP \cite{Radford_CLIP_2021} là biểu diễn cả hai thành phần văn bản và hình ảnh vào cùng một không gian vec-tơ. Như được minh họa tại hình \ref{fig:clip_arch}, mô hình CLIP \cite{Radford_CLIP_2021} được xây dựng bao gồm 2 thành phần chính: một bộ mã hóa văn bản và một bộ mã hóa hình ảnh.

\begin{figure}[htp]
	\centering
	\includegraphics[width=0.8\linewidth]{figures/architecture/CLIP_arch.png}
	\caption{Kiến trúc của mô hình CLIP.}
	\label{fig:clip_arch}
\end{figure}

Với tác vụ tô màu ảnh độ xám có hướng dẫn văn bản, mô hình sử dụng bộ mã hóa văn bản của CLIP \cite{Radford_CLIP_2021} để mã hóa các lời nhắc văn bản thành các vec-tơ, sau đó đưa vào mô hình thông qua cơ chế chú ý chéo.

\subsection{Bộ mã hóa điều kiện}

Để mã hóa điều kiện kiểm soát không gian vào không gian tiềm ẩn, mô hình sử dụng một mạng tích chập nhỏ được huấn luyện chung với quá trình đào tạo mô hình chính. Cấu trúc của mạng mã hóa điều kiện $\mathcal{E}_c$ được thể hiện trong hình \ref{fig:c_encoder}.

\begin{figure}[htp]
	\centering
	\includegraphics[width=0.75\linewidth]{figures/architecture/cond_encoder_arch.png}
	\caption{Cấu trúc mạng mã hóa điều kiện $\mathcal{E}_c$.}
	\label{fig:c_encoder}
\end{figure}

\subsection{Mô hình BLIP}

BLIP \cite{blip} là một mô hình ngôn ngữ - ảnh tiền huấn luyện đa nhiệm. BLIP giải quyết hai hạn chế lớn với hai đề xuất:

\begin{itemize}
	\item Hỗn hợp các cặp bao gồm bộ mã hóa và giải mã đa phương thức (MED): một kiến trúc có khả năng hỗ trợ quá trình tiền huấn luyện đa nhiệm.

	\item Chú thích và lọc: một kỹ thuật tạo dữ liệu mới nhằm tăng cường khả năng học từ các cặp dữ liệu hình ảnh - văn bản nhiễu.
\end{itemize}

\begin{figure}[htp]
	\centering
	\includegraphics[width=\textwidth]{figures/blip/med.pdf}
	\caption{Kiến trúc của hỗn hợp các cặp bộ mã hóa-giải mã đa phương thức MED.}
	\label{fig:med}
\end{figure}

Trong mô hình, nhóm sử dụng mô hình BLIP \cite{blip} để tạo câu mô tả cho ảnh đầu vào, dùng như lời nhắc văn bản cho bộ mã hóa văn bản của CLIP \cite{Radford_CLIP_2021}.

\section{Mô hình tạo sinh ảnh có kết hợp điều kiện không gian}

Mô hình được đề xuất dựa trên mô hình khuếch tán có kết hợp điều kiện được phát triển qua các giai đoạn:

\begin{itemize}
	\item Sự ra đời và phát triển của mô hình khuếch tán trong lĩnh vực tạo sinh ảnh.
	\item Quá trình điều chỉnh nhằm giảm độ phức tạp tính toán bằng cách đưa các tài nguyên huấn luyện vào không gian tiềm ẩn \cite{LDM_Rombach_Blattmann_Lorenz_Esser_Ommer_2022}.
	\item ControlNet \cite{ControlNet} ra đời để kết hợp các điều kiện kiểm soát không gian vào các mô hình khuếch tán đã được huấn luyện sẵn.
\end{itemize}

\subsection{Mô hình khuếch tán}

Mô hình khuếch tán là một lớp mô hình sinh dữ liệu, được thiết kế để mô phỏng các phân phối dữ liệu phức tạp thông qua việc đảo ngược quá trình khuếch tán. Lấy cảm hứng từ vật lý thống kê phi cân bằng \cite{theorical_framework_of_dm}, các mô hình này định nghĩa quá trình sinh dữ liệu như một chuỗi khử nhiễu dần dần.

Quá trình khuếch tán tiến được định nghĩa như sau:

\begin{equation}
	q(\mb{x}_t | \mb{x}_{t-1}) = \mathcal{N}(\mb{x}_t; \sqrt{1-\beta_t}\mb{x}_{t-1}, \beta_t \mb{I})
\end{equation}

Trong đó $\beta_t$ là lịch trình nhiễu tại bước $t$. Mô hình được huấn luyện để dự đoán nhiễu $\boldsymbol{\epsilon}$:

\begin{equation}
	\mathcal{L} = \mathbb{E}_{t,\mb{x}_0,\boldsymbol{\epsilon}} \left[ \| \boldsymbol{\epsilon} - \boldsymbol{\epsilon}_\theta(\mb{x}_t, t) \|^2 \right]
\end{equation}

\subsection{Mô hình khuếch tán trong không gian tiềm ẩn}

Mô hình khuếch tán trong không gian tiềm ẩn \cite{LDM_Rombach_Blattmann_Lorenz_Esser_Ommer_2022} thực hiện quá trình khuếch tán trong không gian tiềm ẩn thay vì không gian điểm ảnh. Quá trình này bao gồm:

\begin{enumerate}
	\item Mã hóa ảnh gốc $\mb{x}$ vào không gian tiềm ẩn: $\mb{z} = \mathcal{E}(\mb{x})$
	\item Thực hiện khuếch tán trong không gian tiềm ẩn
	\item Giải mã từ không gian tiềm ẩn: $\mb{x} = \mathcal{D}(\mb{z})$
\end{enumerate}

Hình \ref{fig:ldm_arch} minh họa kiến trúc của mô hình LDM.

\begin{figure}[htp]
	\centering
	\includegraphics[width=0.9\linewidth]{figures/architecture/LDM_arch.png}
	\caption{Kiến trúc của mô hình khuếch tán trong không gian tiềm ẩn (LDM).}
	\label{fig:ldm_arch}
\end{figure}

\subsection{ControlNet}

ControlNet \cite{ControlNet} cho phép tích hợp các điều kiện không gian vào các mô hình khuếch tán đã được tiền huấn luyện. Kiến trúc của ControlNet được minh họa trong hình \ref{fig:controlnet_arch}.

\begin{figure}[htp]
	\centering
	\includegraphics[width=0.85\linewidth]{figures/architecture/controlnet_arch.png}
	\caption{Kiến trúc của ControlNet.}
	\label{fig:controlnet_arch}
\end{figure}

ControlNet hoạt động bằng cách tạo một bản sao có thể huấn luyện của các khối encoder và middle block của U-Net, trong khi giữ nguyên các tham số của mô hình gốc. Các lớp tích chập khởi tạo bằng không đảm bảo rằng ở giai đoạn đầu của quá trình huấn luyện, ControlNet không ảnh hưởng đến mô hình gốc.


\chapter{Thực nghiệm}
\label{Chapter4}
\textcolor{blue}{Mục này tương đối đơn giản, chỉ cần liệt kê về tập dữ liệu, cài đặt chi tiết cho thực nghiệm, kế hoạch thực nghiệm, kết quả thực nghiệm so với nghiên cứu gốc (baseline).}

Ở mục phân tích kết quả, ngoài việc phân tích về sự cải thiện trên kết quả, cần xem xét các khía cạnh khác như:
\begin{itemize}
    \item Thời gian huấn luyện ra mô hình cuối cùng (100K).
    \item Chi phí của quá trình con (ví dụ, 100 iterations mất bao lâu).
    \item Sự biến thiên của hàm mất mát, độ chính xác (dựa trên log, vẽ biểu đồ đường thì càng tốt).
    \item Phải phân tích các khía cạnh mà mô hình đề xuất bị kém đi, nguyên nhân nào gây ra (do cấu hình chưa tối ưu, hay đó là điều không thể tránh khỏi khi dùng kĩ thuật đó, hay do dữ liệu). Đề xuất một số cách khắc phục các hạn chế đó (dành cho phần hướng nghiên cứu tương lai).
\end{itemize}
\section{Mục tiêu thực nghiệm}

Nhóm tiến hành huấn luyện để tạo ra 4 mô hình cho các mục tiêu khác nhau. Một mô hình được huấn luyện để tô màu cho các trường hợp tổng quát, 3 mô hình còn lại được huấn luyện trên các tập dữ liệu chuyên biệt.

Sau đó nhóm sử dụng các độ đo FID và Colorfulness để so sánh trong tác vụ tô màu tự động và sử dụng thêm CLIP score để so sánh trong tác vụ tô màu bán tự động.

\section{Chi tiết quá trình huấn luyện}

\subsection{Tập dữ liệu huấn luyện}

Nhóm huấn luyện các mô hình trên 4 tập dữ liệu:

\begin{itemize}
    \item \textbf{ImageNet100k:} Tập dữ liệu do nhóm tạo ra bằng cách chọn ngẫu nhiên 100 ngàn ảnh từ tập dữ liệu huấn luyện ImageNet \cite{russakovsky2015imagenetlargescalevisual}. Tất cả ảnh được điều chỉnh về kích thước $512 \times 512$.

    \item \textbf{Fashion Product Images:} Tập dữ liệu bao gồm 44,441 ảnh có kích thước chung $1800 \times 2400$.

    \item \textbf{Furniture Image:} bao gồm 15,000 ảnh của 5 loại nội thất riêng biệt.

    \item \textbf{VN-celeb:} Bộ dữ liệu bao gồm 23105 khuôn mặt của 1020 người.
\end{itemize}

\subsection{Quá trình tiền xử lý dữ liệu}

Quy trình tiền xử lý dữ liệu gồm 3 bước:

\begin{enumerate}
    \item \textbf{Lọc ảnh độ xám:} Lọc ra các ảnh độ xám, chỉ giữ lại ảnh màu. Ảnh được xem là ảnh độ xám khi:

    \begin{equation}
        E(Var(C_i,C_j)) = \frac{1}{3}\sum_{(i,j) \in \{(R,G),(G,B),(B,R)\}} Var(C_i - C_j) < t
    \end{equation}

    Trong đó $t = 12$ là ngưỡng độ xám.

    \item \textbf{Thay đổi độ phân giải:} Tất cả ảnh được điều chỉnh về kích thước $512 \times 512$.

    \item \textbf{Tạo sinh câu mô tả:} Sử dụng mô hình BLIP\cite{blip} để tạo sinh câu mô tả cho ảnh màu.
\end{enumerate}

Bảng \ref{tab:train-dataset} thể hiện số lượng ảnh của 4 tập dữ liệu sau khi tiền xử lý.

\begin{table}[ht]
    \centering
    \begin{tabular}{lcccc}
        \toprule
        Tập dữ liệu & ImageNet100k & Fashion Product & Furniture Image & VN-celeb \\
        \midrule
        Số lượng ảnh & 97,590 & 38,284 & 13,525 & 22,284 \\
        \bottomrule
    \end{tabular}
    \caption{Thống kê số lượng ảnh theo từng tập huấn luyện sau khi tiền xử lý.}
    \label{tab:train-dataset}
\end{table}

\subsection{Kết quả quá trình huấn luyện}

Nhóm trình bày kết quả trực quan của quá trình huấn luyện. Khi sử dụng các lớp tích chập khởi tạo bằng không, ở giai đoạn đầu, các ảnh màu được sinh ra ngẫu nhiên không giống với ảnh độ xám điều kiện. Sau một thời gian học, ở khoảng bước thứ 3100 trở đi, ảnh màu được sinh ra bắt đầu giống với ảnh điều kiện, đây là hiện tượng \textbf{hội tụ đột ngột} được giới thiệu bởi Zhang và các đồng nghiệp \cite{ControlNet}.

\begin{figure}[H]
    \centering
    \begin{tabular}{@{} >{\centering\arraybackslash}m{1.5cm} c @{}}
		\hline
		\rule{0pt}{1.2ex} & \rule{0pt}{1.2ex} \\[-1.0ex]
        \multirow{2}{*}{\footnotesize 2650 bước} & \includegraphics[width=0.5\textwidth]{figures/training/samples-002650.png}        \\
                                   & \includegraphics[width=0.5\textwidth]{figures/training/reconstruction-002650.png} \\ [0.1ex]
        \hline
        \rule{0pt}{1.2ex} & \rule{0pt}{1.2ex} \\[-1.0ex]
        \multirow{2}{*}{\footnotesize \textbf{3200 bước}} & \includegraphics[width=0.5\textwidth]{figures/training/samples-003200.png}        \\
                                   & \includegraphics[width=0.5\textwidth]{figures/training/reconstruction-003200.png} \\ [0.15ex]
		\hline
		\rule{0pt}{1.2ex} & \rule{0pt}{1.2ex} \\[-1.0ex]
        \multirow{2}{*}{\footnotesize 9300 bước} & \includegraphics[width=0.5\textwidth]{figures/training/samples-009300.png}        \\
                                   & \includegraphics[width=0.5\textwidth]{figures/training/reconstruction-009300.png} \\ [0.1ex]
		\hline
    \end{tabular}
    \caption{Chất lượng ảnh được tạo sinh trong quá trình huấn luyện. Ở lân cận bước thứ 3200, hiện tượng \textbf{hội tụ đột ngột} xảy ra.}
    \label{fig:convergence}
\end{figure}

\section{Thử nghiệm mô hình tô màu tổng quát}

\subsection{Tập dữ liệu thử nghiệm}

Để đánh giá độ hiệu quả của mô hình, nhóm sử dụng 3 tập dữ liệu đánh giá:

\begin{itemize}
    \item \textbf{Tập val5k:} chứa 5000 ảnh từ bộ dữ liệu đánh giá của ImageNet \cite{russakovsky2015imagenetlargescalevisual}.
    \item \textbf{Tập ctest:} tập 10000 ảnh được đề xuất bởi Larson và các đồng nghiệp \cite{larsson2017learningrepresentationsautomaticcolorization}.
    \item \textbf{Tập COCO-Stuff:} với 5000 ảnh từ bộ dữ liệu COCO-Stuff \cite{coco-stuff}.
\end{itemize}

\subsection{Kết quả thử nghiệm}

Kết quả thực nghiệm đối với các mô hình tô màu tự động được thể hiện trong bảng \ref{tab:uncond-metrics}. Các điểm số được in đậm thể hiện giá trị tốt nhất, các chỉ số được gạch chân thể hiện chỉ số tốt thứ hai.

\begin{table}[ht]
    \centering
    \resizebox{\textwidth}{!}{
        \begin{tabular}{c |c c | c c | c c}
            \toprule
            Tập dữ liệu & \multicolumn{2}{c|}{ImageNet (val5k)} & \multicolumn{2}{c|}{ImageNet (ctest)} & \multicolumn{2}{c}{COCO-Stuff} \\
            \cmidrule(lr){2-3} \cmidrule(lr){4-5} \cmidrule(lr){6-7}
            Độ đo & FID$\downarrow$ & Colorfulness$\uparrow$ & FID$\downarrow$ & Colorfulness$\uparrow$ & FID$\downarrow$ & Colorfulness$\uparrow$ \\
            \midrule
            CIC \cite{Zhang_Isola_Efros_2016} & 22.0860 & 37.0313 & 12.7651 & 37.5761 & 33.3418 & 37.6487 \\
            UGColor \cite{zhang_real-time_2017} & 15.1777 & 27.0966 & 6.5466 & 27.8122 & 21.4010 & 28.4487 \\
            DeOldify \cite{DeOldify} & 10.5191 & 26.4827 & 4.2143 & 23.1538 & 13.4318 & 28.3779 \\
            DDColor \cite{kang2023ddcolorphotorealisticimagecolorization} & \textbf{5.5726} & \underline{42.8370} & \textbf{2.6294} & 42.9575 & \textbf{7.2718} & 42.2919 \\
            \midrule
            Chúng tôi & 10.4125 & \textbf{44.2871} & 6.8341 & \textbf{44.7854} & 10.4632 & \textbf{45.2589} \\
            \bottomrule
        \end{tabular}
    }
    \caption{Kết quả so sánh các mô hình tô màu tự động trên các tập dữ liệu đánh giá.}
    \label{tab:uncond-metrics}
\end{table}

Kết quả cho thấy, mô hình của nhóm đã vượt qua hầu hết các phương pháp tô màu tiên tiến trước đó trên phương thức tô màu tự động về chỉ số Colorfulness.

\section{Phân tích kết quả}

Hình \ref{fig:results} minh họa một số kết quả tô màu của mô hình đề xuất.

\begin{figure}[htp]
    \centering
    \begin{tabular}{ccc}
        \includegraphics[width=0.3\linewidth]{figures/inputs/ILSVRC2012_val_00000863.JPEG} &
        \includegraphics[width=0.3\linewidth]{figures/main_rs/ILSVRC2012_val_00000863.png} &
        \includegraphics[width=0.3\linewidth]{figures/gt/ILSVRC2012_val_00000863.JPEG} \\
        (a) Ảnh đầu vào & (b) Kết quả tô màu & (c) Ảnh gốc
    \end{tabular}
    \caption{Ví dụ kết quả tô màu. (a) Ảnh độ xám đầu vào. (b) Ảnh được tô màu bởi mô hình đề xuất. (c) Ảnh màu gốc để so sánh.}
    \label{fig:results}
\end{figure}


\chapter{Kết luận}
\label{Chapter5}

Tổng kết lại đóng góp của nghiên cứu, các cải tiến, và hạn chế, kèm theo hướng nghiên tương lai (đã đề cập ở phần \textbf{Phân tích kết quả} của \ref{Chapter4}).

\section{Kết luận}

Trong đề tài này, nhóm đã nghiên cứu và phát triển một mô hình tô màu ảnh độ xám dựa trên mô hình khuếch tán. Nguyên lý thêm nhiễu và khử nhiễu của mô hình khuếch tán, kèm thêm khả năng tùy chỉnh các tham số điều khiển giúp cho mô hình có thể tạo sinh đầu ra đa dạng. Mô hình tiền huấn luyện đã học trên số lượng ảnh đủ lớn, nên tri thức về màu sắc của các đối tượng tự nhiên hay nhân tạo gần như đã được kết xuất.

Việc tinh chỉnh mô hình trên các tập dữ liệu chuyên biệt giúp cho các mô hình chuyên biệt tập trung hơn vào một miền dữ liệu cụ thể, tăng cường chất lượng ảnh được tô màu. Mô hình cũng có khả năng hoạt động trong hai chế độ chính: tô màu tự động hoàn toàn, và tô màu có điều kiện tuân theo hướng dẫn bằng lời nhắc văn bản.

Về mặt dữ liệu, nhóm đã thu thập, xử lý và chuẩn bị các tập dữ liệu phục vụ cho huấn luyện các mô hình tô màu chuyên biệt, bao gồm dữ liệu khuôn mặt, nội thất và thời trang.

\section{Bàn luận}

Mô hình tô màu do nhóm đề xuất đã đạt được kết quả khả quan khi tiến hành đánh giá thực nghiệm trên nhiều tập dữ liệu khác nhau. Tuy nhiên, mô hình cũng vẫn còn gặp một số thách thức như:

\begin{itemize}
	\item Trong những trường hợp mô tả mơ hồ, thiếu ngữ cảnh kết quả vẫn có thể không đúng như mong đợi.

	\item Khả năng hiểu ngôn ngữ phụ thuộc mạnh vào chất lượng bộ mã hóa văn bản.

	\item Khi mô tả văn bản xung đột với đặc trưng hình ảnh, mô hình có thể bị lúng túng giữa việc tin ảnh hay tin văn bản.

	\item Các tập dữ liệu trên các miền chuyên biệt được chuẩn bị chưa quá tốt, dẫn đến kết quả đánh giá của các mô hình chuyên dụng chưa đạt được mong muốn thực tế.
\end{itemize}

Quá trình thêm điều kiện vào mô hình thông qua các bộ mã hóa đặc trưng và các cơ chế như chú ý chéo góp phần rất quan trọng vào việc cải thiện hiệu suất của mô hình huấn luyện. Nghiên cứu có thể được mở rộng bằng cách thay thế hoặc phát triển các thành phần này để có thể tăng khả năng tô màu của mô hình cũng như tích hợp thêm các điều khiển khác.

Kiến trúc kết hợp điều kiện vào các mô hình khuếch tán cho phép các sự thay đổi linh hoạt trong mô hình. Các hướng nghiên cứu tiếp theo có thể mở rộng sang việc tích hợp thêm các điều kiện không gian khác nhằm tạo ra các mô hình mới trong các lĩnh vực liên quan.


% Công trình của tác giả (nếu không có thì comment 02 dòng dưới)
% \addcontentsline{toc}{chapter}{Danh mục công trình của tác giả}
% \include{Appendix/publish}

% In tài liệu tham khảo
\addcontentsline{toc}{chapter}{Tài liệu tham khảo}
\printbibheading[title={Tài liệu tham khảo}]



%\printbibliography

\printbibliography[heading=subbibliography, title={Tiếng Việt}, keyword=Viet, resetnumbers=true]

\DeclareNameAlias{sortname}{last-first}
\DeclareNameAlias{default}{last-first}

\printbibliography[heading=subbibliography, title={Tiếng Anh}, notkeyword=Viet, resetnumbers=1]
% ===================================================================== %
% CHÚ Ý: phải gán lại resetnumbers=số tài liệu tham khảo tiếng Việt + 1 %
% ===================================================================== %

% Phần phụ lục
\appendix

\chapter{Phụ lục}
\label{Appendix1}

% ===================================================================
% VÍ DỤ VỀ PHỤ LỤC
% ===================================================================

\section{Bảng đối chiếu thuật ngữ}

Bảng \ref{tab:vietanh} cung cấp bảng đối chiếu các thuật ngữ Việt-Anh:

\begin{table}[ht]
    \centering
    \caption{Phụ lục đối chiếu Việt Anh}
    \label{tab:vietanh}
    \begin{tabular}{|p{7cm}|p{7cm}|}
        \hline
        \textbf{Tiếng Việt} & \textbf{Tiếng Anh} \\
        \hline
        Trí tuệ nhân tạo & Artificial Intelligence \\
        Học sâu & Deep Learning \\
        Mạng nơ-ron tích chập & Convolutional Neural Network \\
        Mạng nơ-ron & Neural Network \\
        Bộ mã hóa & Encoder \\
        Bộ giải mã & Decoder \\
        Chú ý & Attention \\
        Chú ý chéo & Cross-Attention \\
        Bộ biến đổi & Transformer \\
        Mô hình khuếch tán & Diffusion Model \\
        \hline
    \end{tabular}
\end{table}

\section{Mã nguồn mẫu}

Bạn có thể chèn mã nguồn vào phụ lục:

\begin{lstlisting}[language=Python, caption=Ví dụ mã Python]
def hello_world():
    print("Hello, World!")
    return True

if __name__ == "__main__":
    hello_world()
\end{lstlisting}

\section{Thông tin bổ sung}

Phụ lục có thể chứa các thông tin bổ sung như:
\begin{itemize}
    \item Dữ liệu thực nghiệm chi tiết
    \item Mã nguồn đầy đủ
    \item Các bảng kết quả bổ sung
    \item Tài liệu tham khảo bổ sung
\end{itemize}

% ===================================================================
% LƯU Ý VỀ PHỤ LỤC:
% - Sử dụng \appendix để bắt đầu phần phụ lục
% - Các chương trong phụ lục được đánh số tự động (Phụ lục A, B, ...)
% - Có thể sử dụng \chapter hoặc \section tùy theo cấu trúc
% - Phụ lục thường chứa thông tin bổ sung, không bắt buộc phải đọc
% ===================================================================

% \include{Appendix/appendix2}

\end{document}